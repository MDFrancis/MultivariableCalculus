A \emph{scalar field}\index{scalar field}
is a function that assigns each point in space a
scalar value.  For example, given latitude and longitude $(x,y)$,
the function that outputs the elevation of the surface of the earth
at position $(x,y)$ is a scalar field.  However,
we might ask for a function that when given $(x,y)$ returns
a vector (maybe the velocity of the wind at that point on the earth).
Functions that assign a vector to each point in space
are called \emph{vector fields}\index{vector field}.

Vector fields are key in physics because they model
vector quantities that are non-constant with respect to
space.  For example, force is a vector!  The force due to of gravity
around a point mass is always directed towards the origin---that
means, as you move in space, the direction of the force 
changes.  Suppose there is a point mass at the origin.
We might model the force\footnote{ Despite what is peddled in
Star Wars, this is a real \emph{force field}.} due to this point mass acting on a particle
at position $(x,y,z)$ with the vector field
\[
	\vec F(x,y,z) = \tfrac{1}{(x^2+y^2+z^2)^{3/2}}\mat{-x\\-y\\-z}.
\]
Indeed, the vector $\vec F(x,y,z)$ always points towards the origin,
and $\norm{\vec F(x,y,z)}=\frac{1}{x^2+y^2+z^2}$ is $1/d^2$ where $d$
is the distance from the origin.


\subsection{Notation}
Notationally, $\vec F:\R^n\to\R^m$ indicates that $\vec F$ inputs points or vectors in $\R^n$
and outputs points or vectors in $\R^m$.  Oftentimes in physics $n=m$, but doesn't need to.

When defining a vector field, sometimes it's convenient to use components
and sometimes it's easier to work with vectors.  For example, the force field
$\vec F:\R^2\to\R^3$ above was defined in terms of components as
\[
	\vec F(x,y,z) = \tfrac{1}{(x^2+y^2+z^2)^{3/2}}\mat{-x\\-y\\-z},
\]
but we could have equivalently defined it using vectors by the equation
\[
	\vec F(\vec v) = -\frac{\vec v}{\norm{\vec v}^3}.
\]


\section{Graphing Vector Fields}

Traditionally, a function $f:\R\to\R$ is graphed on a coordinate
plane by picking one axis to be the domain, one
axis to be the range, and then putting a point at each pair $(x,f(x))$.
For a function $g:\R^2\to\R$, we do something similar, except, this time,
the domain is a plane.  But, no matter, we arrange a third axis orthogonal
to a plane and graph points $(x,y,f(x,y))$ in this three-dimensional space.

With vector fields, we cannot graph in the same way.  A vector field
$\vec F:\R^2\to\R^2$ inputs vectors in a plane and outputs vectors in
a plane. To traditionally ``graph'' this function, we would need to
plot points in $\R^2\times\R^2=\R^4$.  Four-dimensional plots are hard to make!
Fortunately, the goal of graphing is rarely to make a faithful representation
of a function---it is to make a visual representation of a function
to aid our thinking.  With this perspective, representing four-dimensional
graphs won't be hard.

If we think of $\vec F:\R^2\to\R^2$ as attaching a vector to each point in
$\R^2$, we might graph $\vec F$ by drawing vectors at each point in $\R^2$.

XXX Figure

This type of plot is called a \emph{quiver plot}\index{quiver plot},
and it will be the main way we visualize vector fields.

\begin{example}
	Graph the vector field $\vec F:\R^2\to\R^2$ defined by $\vec F(x,y)=\mat{x\\y}$.

	XXX Finish
\end{example}

When making a quiver plot, there are two things to consider: (i) how
densely do you draw vectors, and (ii) how do you scale the vectors?
The answer to these two questions depends on the situation---remember,
the goal of plotting a vector field is to visualize what is happening.

Consider the vector field $\vec A:\R^2\to\R^2$ defined by $\vec A(x,y)=\mat{-y\\x}$.
If we plot vectors every $0.5$ units and plot the vectors at the same scale
as the domain, we get the following, hard to follow, figure.

XXX Figure

However, if we plot vectors every $0.5$ units but scale the vectors so they are
$1/4$ the size they'd conventionally be, we get a much more understandable figure.

XXX Figure

We might even decide that the magnitude of each vector is not so important
and totally ignore it in our plot.

XXX Figure with unit vectors


\begin{exercises}
\end{exercises}

\section{The Gradient}

We already have familiarity with one class of vector fields---gradients.
Given a differentiable function $f:\R^n\to\R$, we can always consider the
vector field $\nabla f:\R^n\to\R^n$.  And, $\nabla f$ has some nice geometric
properties.  For instance, $\nabla f$ always points in the direction
of greatest change for $f$.

XXX Figure surface with vector field at base

The connections between $f$ and $\nabla f$ run deep.  Suppose
for a moment that $f:\R^2\to\R$ is affine.  That is, $f(x,y)=\alpha x+\beta y+\gamma$. 
Now, $\nabla f(x,y)=\mat{\alpha\\\beta}$ is
constant.  If $\vec a,\vec b\in\R^2$, we can consider the change in height
of $f$ when moving from $\vec a$ to $\vec b$.  This is,
\[
	f(\vec b)-f(\vec a) = \alpha b_x+\beta b_y - \alpha a_x - \beta a_y
	=\nabla f(\vec a)\cdot (\vec b-\vec a).
\]
Of course this is no surprise---we first encountered
gradients when finding affine approximations of
functions.  Baked into the very core of the gradient is the fact that for a 
differentiable function $h:\R^n\to\R$,
\[
	h(\vec b)-h(\vec a)=\text{change in }h\text{ from }\vec a\text{ to }\vec b
	\approx \nabla h(\vec a)\cdot (\vec b-\vec a),
\]
so long as $\vec a$ and $\vec b$ are close together.  Now, let's apply
some calculus reasoning.

Let $h:\R^n\to\R$ be a differentiable function and let $\vec a,\vec b\in\R^n$
be points that \emph{aren't} close to each other.  Let $\vec a=\vec x_1,\vec x_2,\ldots,\vec
x_m=\vec b$ be a sequence of points so that $\norm{\vec x_{k+1}-\vec x_k}$ is small.
We may think of $\vec x_i$ as a sequence of tiny steps used to get from $\vec a$ to
$\vec b$.  Now, we have
\[
	h(\vec b)-h(\vec a)=\text{change in }h\text{ from }\vec a\text{ to }\vec b
	\approx 
	\sum \nabla h(\vec x_k)\cdot (\vec x_{k+1}-\vec x_k).
\]
If we define $\Delta \vec x_k=\vec x_{k+1}-\vec x_k$, we get an eerily familiar
formula:
\[
	h(\vec b)-h(\vec a)
	\approx 
	\sum \nabla h(\vec x_k)\cdot \Delta \vec x_k.
\]
On the left we have the scalar quantity $h(\vec b)-h(\vec a)$ and on the right
we have a Riemann sum approximation of the vector line integral\index{vector line integral}
of $\nabla h$ along some path starting at $\vec a$ and ending at $\vec b$.  After
taking a limit, this approximation becomes exact.

\begin{theorem}
	\label{THMFTC1}
	Let $h:\R^n\to\R$ be a differentiable function and let
	$\mathcal C\subset \R^n$ be a parameterized, bounded curve 
	which starts at $\vec a\in\R^n$ and ends at $\vec b\in\R^n$.
	Then,
	\[
		h(\vec b)-h(\vec a) = \int_{\mathcal C} \nabla h\cdot \d\vec r,
	\]
	where $\int_{\mathcal C} \nabla h\cdot \d\vec r$ is the vector line
	integral of $\nabla h$ along $\mathcal C$.
\end{theorem}

Theorem \ref{THMFTC1} should remind you of the fundamental theorem of calculus,
which states for a differentiable function $f:\R\to\R$, 
\[
	f(b)-f(a) = \int_a^b f'.
\]
In one dimension, there aren't many ways to get between two points.  In
contrast, there are infinitely many ways to get between two points
in higher dimensions.  Theorem \ref{THMFTC1} says that all these ways
are equivalent, and that $\nabla h$ takes the place of $h'$.

\subsection{Orientation}

As we delve into the realm of vector fields, we need to take note of something
new---orientation\index{orientation}.  In a scalar line integral of $f$ along the curve
$\mathcal C$, we added up the ``height'' of $f$ along the curve $\mathcal C$.
It didn't matter how we traverse $\mathcal C$, and so it
doesn't matter how we parameterize $\mathcal C$ when computing $\int_C f$.  
The area of the graph of $f$ above
$\mathcal C$ depends only on $f$ and $\mathcal C$.  However, things are different for
vector line integrals.

If $\mathcal C$ connects the points $\vec a$ and $\vec b$ and we wish to compute
the amount of work done by the force field $\vec h$ when moving from $\vec a$ to
$\vec b$ along $\mathcal C$, not all parameterizations of $\mathcal C$ will
do.  We must parameterize $\mathcal C$ such that we start at $\vec a$ and
end at $\vec b$.  Thus, the vector integral $\int_{\mathcal C} \vec h\cdot \d\vec r$
depends not only on $\vec h$ and $\mathcal C$, but also on the direction
or \emph{orientation}\index{oriented curve} of $\mathcal C$.

For curves, there are only two orientations, forwards and backwards,
and reversing the orientation of a curve in a vector line integral just multiplies
the result by $-1$.

\begin{example}
	XXX simple example with orientation flipped
\end{example}


The concept of orientation is familiar from single-variable calculus.
After all,
\[
	\int_a^b f=-\int_b^a f.
\]

From now on, if it matters, we will always specify a curve with
its orientation.  If the curve is closed, we will specify its orientation
as \emph{clockwise} or \emph{counterclockwise}.

\subsection{Conservative Vector Fields}

Theorem \ref{THMFTC1} shows that gradients and vector line integrals
interact nicely.  So nicely, in fact, that vector fields coming 
from gradients have a special name.

\begin{definition}[Conservative Vector Field \& Potential Function]
	A vector field $\vec h:\R^n\to\R^n$ is called \emph{conservative}
	if there is some function $f:\R^n\to\R$ so that $\nabla f=\vec h$.
	Such an $f$ is called a \emph{potential function for $\vec h$}.
\end{definition}

We can think of conservative vector fields\index{conservative vector field}
as those having anti-derivatives,
called \emph{potential functions}\index{potential function}\footnote{
In math, if $\nabla f=\vec h$, we say $f$ is a potential function for $\vec h$.
In physics you say that $-f$ is a potential function for $\vec h$---this aligns
more closely with a physicist's notion of \emph{potential energy}.}.  Once
we have a potential function $f$ for a vector field $\vec h$, 
Theorem \ref{THMFTC1} shows that we can compute vector line integrals
of $\vec h$ by evaluating $f$ on the endpoints of our path.  An immediate consequence
is that the integral around a closed path in a conservative vector field is
zero.

\begin{example}
	XXX Example --- work on a hill via complicated path and sub for
	and easy path.
\end{example}

Since conservative vector fields make vector line integrals so easy, it's useful
to determine whether a given vector field is conservative.  The only sure-fire
way to determine a vector field is conservative is to find a potential function.
However, there are several heuristics for showing that a vector field is not
conservative.

\bigskip

Having established that conservative vector fields are useful,
it would be nice to be able to decide if a given vector field is
conservative.  The only sure-fire way to do so is to produce a 
potential function.  However, there methods we can use to quickly
determine if a particular vector field \emph{cannot} be conservative.

The \emph{screening test}\index{screening test} 
is one such method.  Suppose that $\vec F=\mat{F_x\\F_y}$
and that $\vec F$ is conservative.  That means $\vec F=\nabla f$
for some potential function $f:\R^2\to\R$, and so
\[
	F_x = \frac{\partial f}{\partial x}\qquad\text{and}\qquad 
	F_y=\frac{\partial f}{\partial y}.
\]

If $\vec F$ is a \emph{nice} vector field (that is, it comes from a sufficiently differentiable
$f$), then
\[
	\frac{\partial F_x}{\partial y} = 
	\frac{\partial^2 f}{\partial y\partial x} = 
	\frac{\partial^2 f}{\partial x\partial y} = 
	\frac{\partial F_y}{\partial x}. 
\]

This means that if $\frac{\partial F_x}{\partial y} \neq \frac{\partial F_y}{\partial x}$,
we can conclude that $\vec F$ is \emph{not} conservative.

\begin{example}
	Is the vector field $\vec F(x,y)=\mat{x+1\\y+x^2}$ conservative?

	Define $F_x(x,y)=x+1$ and $F_y(x,y)=y+x^2$ so that $\vec F  =\mat{F_x\\F_y}$.
	Now,
	\[
		\frac{\partial F_x}{\partial y} = 0
		\neq 2x = 
		\frac{\partial F_y}{\partial x},
	\]
	and so $\vec F$ is not conservative.
\end{example}

It should be noted that the screening test can only prove vector fields
are not conservative.  It cannot be used to show that they are!

\begin{example}
	Is the vector field
	$\vec F(x,y) = \mat{\frac{-y}{x^2+y^2}\\[4pt]\frac{x}{x^2+y^2}}$ conservative?

	Defining $F_x(x,y)=\frac{-y}{x^2+y^2}$ and $F_y(x,y)=\frac{x}{x^2+y^2}$
	so that $\vec F = \mat{F_x,F_y}$, we compute
	\[
		\frac{\partial F_x}{\partial y} = \frac{y^2-x^2}{(x^2+y^2)^2}
		= 
		\frac{\partial F_y}{\partial x},
	\]
	and so the screening test cannot be used to conclude anything
	about whether or not $\vec F$ is conservative.

	
	However, letting $\mathcal C$ be the circle of radius 1 centered at the
	origin and oriented counter clockwise, we compute $\int_{\mathcal C} \vec F\cdot \d \vec r
	=2\pi$.  If $\vec F$ were a conservative vector field, the 
	vector line integral around any closed path would be $0$.  Thus, we conclude
	that $\vec F$ is not conservative.
	
\end{example}

The screening test works analogously for higher-dimensional vector fields.  One just
needs to test all combinations of mixed partials---if any combination fails, the vector
field is not conservative.

\begin{exercises}
\end{exercises}


\section{Flux and Divergence}

Suppose that the vector field $\vec F(x,y,z)=(4,0,0)$
describes the velocity of water in a river of uniform density.
You cast a $1\times 1$ square net in the river with normal
vector $(1,0,0)$ and ask the following question: how much 
water per unit time is flowing through the net?

XXX Figure

In this situation, it is easy to compute.  Let us assume
the density of the water is $1$.  In this case, since the direction
of flow is orthogonal to the net (because the normal vector is parallel
to the direction of flow), we know
\[
	\text{mass}/\text{unit time} = (\text{net area})
	(\text{water speed})=(1)(4)=4.
\]

If we angle the net slightly, so that it has a normal vector
$(1,1,0)$, computing the amount of water that goes through
the net in a time unit is still not so bad.
\[
	\text{mass}/\text{unit time} = (\text{perceived net area})
	(\text{water speed}).
\]
Here, the \emph{perceived net area} is the area that the net appears to
the water.  Since the net is angled at $45^\circ$ relative to the flow
of the water, the net \emph{appears} smaller, from the perspective of the water.

XXX Figure showing ``apparent'' height

To the water, the net appears to have width $1$ and height $\sqrt{2}/2$, giving
the net a perceived area of $\sqrt{2}/2$.  Now we can compute
\[
	\text{mass}/\text{unit time} = (\text{perceived net area})
	(\text{water speed}) =\tfrac{\sqrt 2}{2}(4)=2\sqrt{2}.
\]

Equivalently, we could have thought about water from the perspective of the net.
We know
\[
	\text{mass}/\text{unit time} = (\text{net area})
	(\text{water speed normal to net}),
\]
and computing the velocity of water normal to the net can be done with a dot product.
In particular,
\[
	\text{water speed normal to net}=\Proj_{\vec n}\vec v = \mat{2\\2\\0},
\]
where $\vec n=(1,1,0)$ is the normal vector for the net and $\vec v=(4,0,0)$
is the velocity vector of the water.  Now
\begin{align*}
	\text{mass}/\text{unit time} &= (\text{net area})
	(\text{water speed normal to net}) \\
	&= (1)\norm*{\mat{2\\2\\0}}=2\sqrt{2},
\end{align*}
which is the same quantity we computed before.

Measuring how much a vector field ``flows'' through a surface $\mathcal S$
is called the \emph{flux}\index{flux}\footnote{
The term \emph{flux} comes from \emph{fluxus},
which means ``flow'' in Latin.
} of the vector field through $\mathcal S$, and comes up in applications
ranging from fluid mechanics to electro-magnetism and relativity gravity.

\subsection{Computing Flux}

Given constant vector fields and flat surfaces, flux is not difficult to compute.
However, the vector fields we encounter in nature are rarely constant and
the surfaces are rarely flat.  But, the ideas of calculus are here
to help us!

By now, we are familiar with the idea of chopping things up into little pieces,
approximating each piece, adding them together again, and taking a limit.  Finding
the flux of a vector field through an arbitrary surface follows this pattern.

Let $\vec F:\R^3\to\R^3$ be a smooth vector field and let $\mathcal S$ be 
a surface\footnote{ An orientable surface, to be precise.}.
Let $\Delta A$ represent a tiny patch of the surface $\mathcal S$.  We can then say
\[
	\Flux_{\mathcal S} \vec F = \sum_{\Delta A} \Flux_{\Delta A} \vec F,
\]
where $\Flux_A \vec B$ means the flux of the vector field $\vec B$ through the
surface $A$.  This looks a lot like a setup for a surface integral!  Let's make
the connection exact.

Let $\vec p:\R^2\to\mathcal S$ be a parameterization of the surface $\mathcal S$,
and let $\Delta A_{(x_0,y_0)}=\vec p([x_0,x_0+\Delta x]\times [y_0,y_0+\Delta y])$
be a tiny sector of $\mathcal S$ with ``lower-left corner'' at $\vec p(x_0,y_0)$.
As long as $\Delta x$ and $\Delta y$ are small, $\Delta A_{(x_0,y_0)}$ will be
small.

Now, let's approximate.  Since $\Delta A_{(x_0,y_0)}$
is small, we could approximate $\vec F$ near $\Delta A_{(x_0,y_0)}$
by the constant vector 
field $\vec F_{(x_0,y_0)}(x,y) = \vec F\circ \vec p(x_0,y_0)$.  Further,
if $\Delta x$ and $\Delta y$ are small enough, we can approximate $\Delta A_{(x_0,y_0)}$
by a tiny parallelogram coming from the 
canonical parameterization of the tangent plane to $\mathcal S$ at $\vec p(x_0,y_0)$.

Recall, given the parameterization $\vec p$,
the canonical parameterization of the tangent plane to $\mathcal S$ at $\vec p(x_0,y_0)$
is
\[
	\vec P(t,s) = t \frac{\partial \vec p}{\partial x}(x_0,y_0) + 
	s\frac{\partial \vec p}{\partial y}(x_0,y_0)
	+\vec p(x_0,y_0).
\]
Since $\vec P$ is a canonical parameterization, if $\Delta x$ and $\Delta y$
are small, then 
\[
\Delta A_{(x_0,y_0)}=\vec p([x_0,x_0+\Delta x]\times [y_0,y_0+\Delta y])
\approx \vec P([x_0,x_0+\Delta x]\times [y_0,y_0+\Delta y]).
\]
Further, $\vec P([x_0,x_0+\Delta x]\times [y_0,y_0+\Delta y])$ is a parallelogram
with normal vector
\[
	\vec n = \frac{\partial \vec p}{\partial x}(x_0,y_0) \times
	\frac{\partial \vec p}{\partial y}(x_0,y_0)
\]
and area $\norm{\vec n}$.

We have developed several approximations.  Before we put them all together, let's
give them simple names.  Fix $x_0$ and $y_0$,
and let $\Delta A=\Delta A_{x_0,y_0}$;
let $Q(x,y) = \vec F\circ \vec p(x_0,y_0)$ be a constant
vector field approximating $\vec F$; and, let 
$\Delta R=\vec P([x_0,x_0+\Delta x]\times [y_0,y_0+\Delta y])$ be a parallelogram
approximating $\Delta A$.  We then have, so long as $\Delta x$ and $\Delta y$ are small
\begin{align*}
	\Delta A &\approx \Delta R\\
	\vec F &\approx \vec Q.
\end{align*}
Putting this all together, 
\begin{align*}
	\Flux_{\Delta A}\vec F
	\approx \Flux_{\Delta A}\vec Q 
	\approx \Flux_{\Delta R}\vec Q
\end{align*}

Now, $\tfrac{\vec n}{\norm{\vec n}}$ is a unit normal vector to $\Delta R$,
and so
\[
	\Flux_{\Delta R}\vec Q = 
	\norm{\vec n}\tfrac{\vec n}{\norm{\vec n}}\cdot \vec Q
	=\vec n\cdot \vec Q.
\]
Substituting in for $\vec n$ and $\vec Q$, we have
\begin{equation}
	\label{EQFLUXAPPROX}
	\Flux_{\Delta A}\vec F
	\approx 
	\left(\frac{\partial \vec p}{\partial x}(x_0,y_0) \times
	\frac{\partial \vec p}{\partial y}(x_0,y_0)\right)
	\cdot \Big(\vec F\circ \vec p(x_0,y_0)\Big ).
\end{equation}

We worked pretty hard to get Equation \eqref{EQFLUXAPPROX}, but
let's remind ourselves that the idea was simple.  We approximated
a changing vector field by a constant one and a curvy shape by a flat one.
The rest of the formula comes from exploiting the relationship between
the cross product and the area of a parallelogram and the computation
of the flux of a constant vector field through a flat surface.





\section{Circulation and Curl}
