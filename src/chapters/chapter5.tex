\section{Multivariable Integrals}

Fundamentally, an integral is the result of chopping a region up into tiny
pieces, adding all the pieces up again, and taking a limit as the size of
the tiny pieces goes to zero.  Thus far, the domain of this procedure---the
thing we chop into tiny pieces---has been a line or curve.  We will now consider
integrating over multi-dimensional domains.

Consider the motivating example of finding the area of a region in the plane.
Let $\mathcal R\subseteq \R^2$ be the region below the line $y=1$ and above the
curve $y=x^2$.  We wish to find the area of $\mathcal R$.

Using the usual calculus strategy, we will chop $\mathcal R$ up into little rectangles
of width $\Delta x$ and height $\Delta y$.  Then
\[
	\text{area of }\mathcal R \quad\approx \sum_{\text{tiny rectangles}}\text{area of tiny rectangle}
\]
and
\[
	\text{area of }\mathcal R =\lim_{\Delta x,\Delta y\to 0} \sum_{\text{tiny rectangles}}\text{area of tiny rectangle}.
\]

XXX Figure

Using integral notation, we would write
\[
	\text{area of }\mathcal R=\int_{\mathcal R} \d A.
\]
Here $\d A$ represents a ``tiny area,'' the subscript $\mathcal R$ represents the region of integration,
and the integral sign means we're adding things up.  In this case, we're finding area, so $\d A=1\d A$ is
exactly what we're adding up.  In other situations we'll be adding up more complicated functions.

This is all well and good, but how do we actually \emph{find} the area.  To do this, we'll
need to convert $\int_{\mathcal R}\d A$ into a more traditional-looking integral---one that we
know how to evaluate.

Let's write down our sum more carefully.  We need to sum over all tiny rectangles that fit inside
$\mathcal R$.  To do so, we can take a systematic approach: let's sum all the rectangles
in a column first and then sum up all the columns.  The lower left corner of all 
rectangles in a single column share a common $x$-coordinate.  Consider
the column with lower left corner at $(x_0,?)$.  Counting, we see there
are approximately $(1-x_0^2)/\Delta y$ rectangle in this column.  Further, there are approximately
$2/\Delta x$ columns.  Therefore,
\[
	\text{area of }\mathcal R\approx 
	\sum_{i=1}^{2/\Delta x}\quad\sum_{j=1}^{(1-(i\Delta x)^2)/\Delta y} \Delta y\Delta x.
\]
That sum is really hard to parse, so we'll write it another way.
\[
	\text{area of }\mathcal R\approx
	\sum_{x_0=-1,-1+\Delta x,-1+2\Delta x,\ldots,1}
	\quad
	\sum_{y_0=x_0^2,x_0^2+\Delta y,x_0^2+2\Delta y,1}
	\Delta y\Delta x.
\]
This is still hard to read, but it's looking more like an integral.  The inner sum
is adding up things from $y_0=x_0^2$ to $y_0=1$ and the outer sum is adding up
things from $x_0=-1$ to $x_0=1$.  Upon taking a limit, we may rewrite this as an
integral, giving
\begin{equation}
	\label{EQITERATEDINT}
	\text{area of }\mathcal R\quad
	=\quad
	\int_{\mathcal R}\d A \quad =\quad \int_{x=-1}^{x=1}\int_{y=x^2}^{y=1} \d y\d x.
\end{equation}
Now, we should take a moment to make sure we understand what we've just written.  The
right side of Equation \eqref{EQITERATEDINT} is an \emph{iterated integral}\index{iterated integral}.
That is,
\[
	\int_{x=-1}^{x=1}\int_{y=x^2}^{y=1} \d y\d x
	\quad 
	=
	\quad \int_{x=-1}^{x=1}\left(\int_{y=x^2}^{y=1} \d y\right)\d x
\]
and so the integral with respect to $y$ \emph{must} be done before the integral with respect to $x$.
To be clear, $\d y$ and $\d x$ are \emph{not} being multiplied.  However, $\Delta y$ and $\Delta x$ 
\emph{were} being multiplied in our sum expression.  What happened?  The answer is some slight of
hand.  We can write a more complete list of steps:\footnote{ Even still, we skipped steps.
In particular, we split up $\lim_{\Delta x,\Delta y\to 0}$ into two limits 
$\lim_{\Delta x\to 0}\lim_{\Delta y\to 0}$.  Then, we exchanged a limit and a sum.  These
steps require justification, but we won't trouble ourselves with that here.}
\[
	\lim_{\Delta x,\Delta y\to 0} \sum_{x_i}\sum_{y_i} \Delta y\Delta x
	=
	\lim_{\Delta x,\Delta y\to 0} \sum_{x_i}\left(\sum_{y_i} \Delta y\right)\Delta x
	=\int_{x=-1}^{x=1}\left(\int_{y=x^2}^{y=1} \d y\right)\d x.
\]

Now we can evaluate this iterated integral to conclude
\[
	\text{area of }\mathcal R
	\quad
	=\quad
	\int_{\mathcal R}\d A \quad=\quad 
	\int_{x=-1}^{x=1}\left(\int_{y=x^2}^{y=1} \d y\right)\d x \quad=\quad
	\tfrac{4}{3}.
\]
But, there was another way we could have divided up our original sum.  We could have summed
along rows first and then summed up each row.  Working from this approach, we see
\[
	\lim_{\Delta x,\Delta y\to 0} \sum_{y_i}\sum_{x_i} \Delta x\Delta y
	=
	\lim_{\Delta x,\Delta y\to 0} \sum_{y_i}\left(\sum_{x_i} \Delta x\right)\Delta y
	=\int_{y=0}^{y=1}\left(\int_{x=-\sqrt{y}}^{x=\sqrt{y}} \d x\right)\d y.
\]
Computing this integral, we again get $4/3$, as expected.

We can use very similar reasoning to turn integrals of other functions
over regions into iterated integrals that we already know how to evaluate.

\begin{example}
	Let $f(x,y) = x^2 + y^2$,
	and let $\mathcal D$ be the region between the
	parabola $x = y^2$ on the left and the line $x = y + 2$
	on the right.  These curves intersect when $y^2=y+2$.  In
	other words, the curves intersect when $y=2$ or $y=1$.  Thus,
	$\mathcal D$ 
	also lies between $y = -1$ below and $y = 2$
	above.   

\begin{center}
	\usetikzlibrary{patterns,decorations.pathreplacing}
	\begin{tikzpicture}
		\coordinate (A) at (1,1);
		\coordinate (B) at (3,2);
		\begin{axis}[
		    anchor=origin,
		    disabledatascaling,
		    xmin=-1,xmax=5,
		    ymin=-1,ymax=2,
		    x=1cm,y=1cm,
		    grid=both,
		    grid style={line width=.1pt, draw=gray!10},
		    %major grid style={line width=.2pt,draw=gray!50},
		    axis lines=middle,
		    minor tick num=0,
		    enlargelimits={abs=0.5},
		    axis line style={latex-latex},
		    ticklabel style={font=\tiny,fill=white},
		    xlabel style={at={(ticklabel* cs:1)},anchor=north west},
		    ylabel style={at={(ticklabel* cs:1)},anchor=south west}
		]

	    	\addplot[name path=f,no marks,mypink,thick,domain=-1:2, samples=25, draw=none] ({x*x},{x});
	    	\addplot[no marks,mypink,thick,domain=-2:3, samples=25] ({x*x},{x});
	    	\addplot[name path=g, no marks,green!50!black,thick,domain=-2:5, samples=2] ({x+2},{x});
		\addplot [thick, color=blue, fill=blue, fill opacity=0.05] fill between [of=f and g, split];
		\draw (1,0) node[above] {$\mathcal D$};
		\draw [mypink,fill] (A)  node [above left] {$x=y^2$};
		\draw [green!50!black,fill] (3,1)  node [below right] {$x=y+2$};

		\end{axis}
	\end{tikzpicture}
\end{center}

	Thus
	\begin{align*}
		\int_{\mathcal D} f\, \d A &= 
		\int_{y=-1}^{y=2}\left(\int_{x = y^2}^{x=y+2} x^2 + y^2\, \d x\right) \d y \\
	   &=\int_{-1}^2 \left( \left.\frac{x^3}3 + xy^2\right|_{x=y^2}^{x=y+2}\right) \d y \\
	   &=\int_{-1}^2 \left(\frac {(y + 2)^3}3 + (y + 2)y^2 -
	    \frac{y^6}3 - y^4 \right) \d y \\
	   &=\left. \left(\frac {(y + 2)^4}{12} + \frac{y^4}4 + \frac{2y^3}3 -
	    \frac{y^7}{21} - \frac{y^5}5 \right)\right|_{-1}^2 \\
	   &=\tfrac{256}{12} + \tfrac{16}4 + \tfrac{16}3 - \tfrac{128}{21} - \tfrac{32}5
	     - \tfrac 1{12} - \tfrac 14 + \tfrac 23 - \tfrac 1{21} - \tfrac 15 \\
	   &=\tfrac{639}{35} \approx 18.26.
	\end{align*}
	Note that the region is also bounded vertically by graphs, so in
	principle the integral could be evaluated in the other order.
	However, there is a serious hindrance to trying this.
	The top graph is that of  $y = h_{\text{top}}(x) = \sqrt x$,
	 but the bottom
	graph is described by two different formulas depending on what $x$
	is.  It is a parabola to the left of the point $(1,-1)$ and a line
	to the right of that point, so  
	\[
		h_{\text{bot}}(x) = \begin{cases}
			-\sqrt{x} & \text{ if }0\leq x\leq 1\\
			x-2 &\text{ if }1 < x\leq 4
			\end{cases}.
	\]

\begin{center}
	\usetikzlibrary{patterns,decorations.pathreplacing}
	\begin{tikzpicture}
		\coordinate (A) at (1,1);
		\coordinate (B) at (3,2);
		\begin{axis}[
		    anchor=origin,
		    disabledatascaling,
		    xmin=-1,xmax=5,
		    ymin=-1,ymax=2,
		    x=1cm,y=1cm,
		    grid=both,
		    grid style={line width=.1pt, draw=gray!10},
		    %major grid style={line width=.2pt,draw=gray!50},
		    axis lines=middle,
		    minor tick num=0,
		    enlargelimits={abs=0.5},
		    axis line style={latex-latex},
			xticklabels={,,},
			yticklabels={,,}
		]

	    	\addplot[name path=f,no marks,mypink,thick,domain=-1:2, samples=25, draw=none] ({x*x},{x});
	    	\addplot[no marks,mypink,thick,domain=0:3, samples=25] ({x*x},{x});
	    	\addplot[no marks,blue,thick,domain=-2:0, samples=25] ({x*x},{x});
	    	\addplot[name path=g, no marks,green!50!black,thick,domain=-2:5, samples=2] ({x+2},{x});
		\addplot [thick, color=blue, fill=blue, fill opacity=0.05] fill between [of=f and g, split];

		\addplot[color=gray, dashed, thick, domain=-3:3] ({1},{x});
		\draw (1.2,0) node[above right] {$\mathcal D$};
		\draw [mypink,fill=white] (A)  node [above left] {$y=\sqrt{x}$};
		\draw [blue,fill=white] (.5,-1)  node [left] {$y=-\sqrt{x}$};
		\draw [green!50!black,fill] (3,1)  node [below right] {$y=x+2$};

		\end{axis}
	\end{tikzpicture}
\end{center}

	(The $x$-values at the relevant points are determined from the corresponding
	$y$-values which were calculated above.)  That means to actually
	compute the integral we must decompose the region
	$\mathcal D$ into two subregions meeting along the line $x = 1$ and treat each
	one separately.   Doing so,
	\[
		\int_{\mathcal D} f\, \d A
	=
	\int_{x=0}^{x=1}\left(\int_{y = -\sqrt x}^{y= \sqrt x} x^2 + y^2\, \d y\right) \d x 
	+
	\int_{x=1}^{x=4}\left(\int_{y = x-2}^{y= \sqrt x} x^2 + y^2\, \d y\right) \d x .
	\]
	You should work out the two iterated integrals on the right just to check
	that their sum gives the same answer as above.

\end{example}

\subsection{Integral Notation}

In single-variable calculus, you used the notation $\tfrac{\d}{\d x}f(x)=\tfrac{\d f}{\d x}(x)$ 
to represent the derivative
of the function $f$ and $\int_{a}^b f(x)\d x$ to represent the integral from $a$ to $b$ of the function
$f$.  This notation was invented by Gottfried Leibniz in the 1600s and was inspired by geometric thinking.
Though the infinitely large and infinitely small need to be handled with mathematical care,
the idea that ``$\d x$'' represents and infinitesimally small change in the variable $x$ gives
rise to fruitful intuitions.

We will extend on this idea.  Consider $f:\R^n\to\R$ and a region $\mathcal R\subseteq \R^n$.
We will write
\[
	\int_{\mathcal R} f\,\d V
\]
to notate \emph{the integral of $f$ over the region $\mathcal R$ with respect to volume}.
If $f:\R^2\to\R$, we may write $\d A$ in place of $\d V$, since a two-dimensional volume
is traditionally called an area.  If the context is clear, we might even omit the $\d V$
entirely, opting to write
\[
	\int_{\mathcal R} f.
\]

This notation can replace single-variable calculus notation.  For example
\[
	\int_a^b f(x)\,\d x = \int_{[a,b]} f.
\]
As always, the purpose of notation is to facilitate our thinking, not to be a formal grammar\footnote{
	Sometimes notation's dual purpose is to be a formal grammar.  For instance,
	most programming languages could be though of as very stringent 
	systems of notation.}, and so you should use whatever notation
most clearly conveys your thoughts.  For us, $\int_{\mathcal R}f$ will often suffice.

When we write iterated integrals, we will omit parenthesis.  For example, we write
\[
	\int \left(\int f(x,y)\,\d x\right)\d y\qquad\text{as}\qquad \int \int f(x,y)\,\d x\d y,
\]
and to keep ourselves from getting the bounds of integration mixed up, we will label
the bounds of integration for iterated integrals with the variable they bound.  For example,
we write
\[
	\int_{y=a}^{y=b}\int_{x=c}^{x=d} f(x,y)\,\d x\d y\qquad\text{instead of}\qquad
	\int_{a}^{b}\int_{c}^{d} f(x,y)\,\d x\d y.
\]
This will prevent us from becoming confused when we swap the order of integration.

Finally, it is worth being aware of other notation that you might encounter.  For example,
some textbooks write
\[
	\iint f\,\d A\qquad\text{and}\qquad \iiint f\,\d V
\]
for integrals with respect to area and volume.  The motivation being that the number of integral
signs should match the number of dimensions you're integrating over.

\subsection{Iterated Integrals}

We've already seen how an integral over a region can be turned into an iterated
integral in multiple ways.  We will turn this into a technique for evaluating
some difficult iterated integrals.  Namely, if we are given an iterated integral,
we can try to convert it into an integral over a region, and then convert it back to a different
iterated integral.

\begin{example}
	Consider the iterated integral
	\[
		\int_{x=0}^{x=1}\int_{y=x}^{y=1} \frac{\sin y}{y}\d y \d x.
	\]
	This is the iterated integral obtained from the double integral
	of the function $f(x,y) = \sin y/y$ for the triangular
	region $\mathcal D$ contained
	between the vertical lines  $x = 0, x = 1$, the line
	$y = x$ below, and the line $y = 1$ above.

	XXX Figure

	   The indefinite
	integral (anti-derivative) 
	\[
	     \int \frac{\sin y}y dy
	\]
	{\it cannot be expressed in terms of known elementary functions}.
	(Try to integrate it or look in an integral table if you don't
	believe that.)   Hence, the iterated integral cannot be evaluated
	by anti-derivatives.  However, the triangular region may be
	described just as well by bounding it horizontally by graphs:
	it lies between $y = 0$ and $y = 1$ and for each $y$ between

	XXX Figure

	$x = 0$ and $x = y$.  
	 Thus, the double integral can be evaluated
	from the  iterated integral
	\begin{align*}
		\int_{y=0}^{y=1}\int_{x=0}^{x=y}\frac{\sin y}y\d x\d y &=
		\int_0^1\, \left. \frac{\sin y}y x\right|_{x=0}^{x=y}\, \d y  \\
		&= \int_0^1\,  \frac{\sin y}y y\, \d y
		= \int_0^1\ \sin y\, \d y  \\
		&= \left. -\cos y\right|_0^1 = 1 - \cos 1.
	\end{align*}

	Note that in order to set up the iterated integral in the other
	order, \emph{we had to draw a diagram} and work directly from
	that.   \emph{There are no algebraic rules which will allow you
	to switch orders without using a diagram.}
\end{example}

Swapping the order of integration is a neat trick, and if an
iterated integral comes from the integral of some function over a
region, then we can always swap the order of integration\footnote{ Depending
on how we define things, this may be a circular argument.  In our definition
of an integral over a region, we had to sum up a bunch of tiny areas in an unspecified
order.  We might \emph{define} this to make sense only if every possible order we could
sum results in the same number.}.  However, if we're given an iterated integral,
a priori, we don't know that we can swap the order of integration.

\begin{example}
	Let $f(x,y)=\frac{x^2-y^2}{(x^2+y^2)^2} = 
	-\tfrac{\partial}{\partial x}\left(\tfrac{\partial}{\partial y}\arctan(y/x)\right)$.  Now,
	\[
		\int_{x=0}^{x=1}\int_{y=1}^{y=1} f(x,y)\,\d y\d x = \frac{\pi}{4}
	\]
	and
	\[
		\int_{y=0}^{y=1}\int_{x=1}^{x=1} f(x,y)\,\d x\d y = -\frac{\pi}{4}.
	\]
	The order of integration matters for $f$ and so the iterated integrals $\iint f\, \d x\d y$
	and $\iint f\,\d y\d x$ cannot have come from the integration of $f$ over a region---an
	integral of $f$ over a region cannot be reasonably defined.
\end{example}

For us, it will be uncommon to come across iterated integrals where the order of integration
cannot be changed.  However, the following theorem from analysis gives us explicit conditions
for when it is okay.

\begin{theorem}[Fubini's Theorem]
	Let $f:\R^2\to R$ and suppose that 
	\[
		\int_{x=a}^{x=b}\int_{y=c}^{y=d} \abs{f(x,y)}\,\d y\d x < \infty.
	\]
	Then,
	\[
		\int_{x=a}^{x=b}\int_{y=c}^{y=d} f(x,y)\,\d y\d x =
		\int_{y=c}^{y=d}\int_{x=b}^{x=a} f(x,y)\,\d x\d y.
	\]
\end{theorem}\index{Fubini's Theorem}

\begin{exercises}
\end{exercises}

\section{The Volume Form}

Let $f:\R^2\to \R$ and $\mathcal D\subseteq \R^2$ be the unit disk centered at the origin.
If we wanted to compute $\int_{\mathcal D} f\,\d A$, we could split this up into an iterated
integral like so:
\[
	\int_{\mathcal D} f\,\d A = \int_{x=-1}^{x=1}\int_{y=-\sqrt{1-x^2}}^{y=\sqrt{1-x^2}} f(x,y)\,\d y\d x.
\]
However, the thought of integrating square roots is painful.  Especially, when $\mathcal D$ has
a simple description in polar coordinates.  If we could use polar coordinates, our iterated
integral might look more like $\int_{\theta=0}^{\theta=2\pi}\int_{r=0}^{r=1} f$.  Much less intimidating!

Let's see if we can figure out how to integrate in polar coordinates.  The principle should be the same.
If we divide $\mathcal D$ into tiny \emph{sectors}, then
\[
	\int_{\mathcal D} f\,\d A =\lim_{\text{sector size}\to 0} \sum_{\text{sectors}\ \in\ \mathcal D} (\text{sector
	area})f(\text{sector location}).
\]
When we previously used iterated integrals to compute integrals over regions, we chopped 
our region into little rectangles of width $\Delta x$ and height $\Delta y$.  This meant
that the area of each sector was $\Delta x\Delta y$, and so we had the formula
\[
	\int_{\mathcal D} f\,\d A =\lim_{\text{sector size}\to 0} \sum_{\text{sectors}\ \in\ \mathcal D} 
	(\Delta x\Delta y) f(x,y).
\]
For polar coordinates\index{polar coordinates}, 
the natural way to chop up $\mathcal D$ is not into rectangles, but
into \emph{annular sectors}.

XXX Figure comparing rectangular chopping and annular sector chopping.

When we compute the exact area of the annular sector between the curves
$\theta=\theta_0$, $\theta=\theta_0+\Delta\theta$, $r=r_0$, and $r=r_0+\Delta r$, 
we get 
\[
	\text{annular sector area} = r_0\Delta r\Delta \theta + \tfrac{1}{2}\Delta\theta(\Delta r)^2.
\]
The first thing to note is that the quantity $r_0$ shows up in the area formula.  
This makes sense, because if you fix two $\theta$ values, the further ``out'' (i.e., the larger
$r_0$ is) you are, the larger the area.  This didn't happen in rectangular coordinates---all the rectangular
sectors had exactly the same area.

Now, just as before, we can write a sum expression for our integral.
\begin{equation}
	\label{EQPOLARSUM}
	\int_{\mathcal D} f\,\d A =\lim_{\Delta r,\Delta \theta\to 0}
	\sum_{r=0,\Delta r,2\Delta r,\ldots,1}\ \ \sum_{\theta=0,\Delta\theta,
	2\Delta\theta, \ldots, 2\pi} \Big(
		r\Delta r\Delta \theta + \tfrac{1}{2}\Delta\theta(\Delta r)^2
	\Big)f(r,\theta)
\end{equation}
However, it isn't clear how to rewrite Equation \eqref{EQPOLARSUM} as an iterated integral
for easy evaluation.

\subsection{Order Analysis}
We can rewrite Equation \eqref{EQPOLARSUM} into one that looks more like a Riemann sum by
doing some \emph{order analysis}\index{order analysis}.  Order analysis allows us to determine
with precision
which parts of an expression become negligible and which parts are non-negligible.  Often
times order analysis will show that we can ignore or erase large parts of a formula and still
get the correct answer in the end.


Consider the right hand side of Equation \eqref{EQPOLARSUM}.
\[
	\lim_{\Delta r,\Delta \theta\to 0}
	\sum_{r=0,\Delta r,2\Delta r,\ldots,1}\ \ \sum_{\theta=0,\Delta\theta,
	2\Delta\theta, \ldots, 2\pi} \Big(
		r\Delta r\Delta \theta + \tfrac{1}{2}\Delta\theta(\Delta r)^2
	\Big)f(r,\theta)
\]
In this expression, both $\Delta r$
and $\Delta \theta$ are tending towards $0$ and so 
\[
	r\Delta r\Delta \theta + \tfrac{1}{2}\Delta\theta(\Delta r)^2\to 0.
\]
But, we don't expect the entire limit to go to zero because as $\Delta\theta$ and $\Delta r$
get smaller, each of the two sums get more and more terms.  The outer sum will have roughly
$1/\Delta r$ terms and the inner sum will have roughly $1/\Delta \theta$ terms\footnote{
We say roughly because the number of terms involved is always an integer and the quantities
$1/\Delta r$ and $1/\Delta \theta$ may not be integers.
}.  
Now, $\sum_{i=1}^n K=nK$ if $K$ is a constant.  The inside of our sums are not constant, but if
they were bounded, we could replace them with constants and apply squeeze-theorem arguments to them.
For order analysis, since we are only interested in whether a term
\emph{must} go to zero, we bypass the detailed squeeze theorem arguments 
by just imagining all terms inside our sum are constant and seeing what happens.
To proceed, we first split up
our sum to figure out which parts are negligible.
\[
			\sum_{r=0,\Delta r,2\Delta r,\ldots,1}\ \ \sum_{\theta=0,\Delta\theta,
	2\Delta\theta, \ldots, 2\pi} \Big(
		r\Delta r\Delta \theta + \tfrac{1}{2}\Delta\theta(\Delta r)^2
		\Big)f(r,\theta)
\]
\[
	\phantom{XXXXXXXX}=\sum_{\theta}\sum_{r} r\Delta r\Delta \theta f(r,\theta)
	+ 
		\sum_{\theta}\sum_{r} \tfrac{1}{2}(\Delta r)^2\Delta \theta f(r,\theta)
\]
Doing order analysis on the first part of the sum, we see
\begin{align*}
	\lim_{\Delta r,\Delta\theta\to 0}
	\sum_{\theta}\sum_{r} r\Delta r\Delta \theta f(r,\theta)
	&\sim
	\lim_{\Delta r,\Delta\theta\to 0}
	\frac{1}{\Delta r}\left(\frac{1}{\Delta \theta} r\Delta r\Delta\theta f(r,\theta)\right)\\
	&=rf(r,\theta),
\end{align*}
where $\sim$ means \emph{order equivalent}\footnote{ There is a formal theory of
order equivalence and a slew of specialized notations.  We don't need those formalities
right now.}.  Since this term is not order-equivalent to zero, it is non-negligible, and so 
we must keep it.  

Analyzing the other component of the sum, we see a different story:
\begin{align*}
	\lim_{\Delta r,\Delta\theta\to 0}
	\sum_{\theta}\sum_{r} \tfrac{1}{2}(\Delta r)^2\Delta \theta f(r,\theta)
	&\sim
	\lim_{\Delta r,\Delta\theta\to 0}
	\frac{1}{\Delta r}\left(\frac{1}{\Delta \theta} \tfrac{1}{2}(\Delta r)^2\Delta\theta f(r,\theta)\right)\\
	&=\lim_{\Delta r,\Delta\theta \to 0}\tfrac{1}{2}\Delta rf(r,\theta) = 0,
\end{align*}
and so this part of the sum \emph{is} negligible.  We may therefore conclude,
\[
	\lim_{\Delta r,\Delta \theta\to 0}
	\sum_{r=0,\Delta r,2\Delta r,\ldots,1}\ \ \sum_{\theta=0,\Delta\theta,
	2\Delta\theta, \ldots, 2\pi} \Big(
		r\Delta r\Delta \theta + \tfrac{1}{2}\Delta\theta(\Delta r)^2
	\Big)f(r,\theta)
\]
\[
	=\lim_{\Delta r,\Delta \theta\to 0}
	\sum_{r=0,\Delta r,2\Delta r,\ldots,1}\ \ \sum_{\theta=0,\Delta\theta,
	2\Delta\theta, \ldots, 2\pi} 
		r\Delta r\Delta \theta f(r,\theta),
\]
which is looking a lot more line a Riemann sum that can be converted into an iterated integral.
In fact, if $f$ is a reasonable function (for example, one where Fubini's theorem would hold),
\[
	\lim_{\Delta r,\Delta \theta\to 0}
	\sum_{r=0,\Delta r,2\Delta r,\ldots,1}\ \ \sum_{\theta=0,\Delta\theta,
	2\Delta\theta, \ldots, 2\pi} 
		r\Delta r\Delta \theta f(r,\theta)
		=\int_{\theta=0}^{\theta=2\pi}\int_{r=0}^{r=1}f(r,\theta)r\d r\d \theta.
\]

Order analysis can be very powerful, but we should take note about the assumptions we made.
First off, for the order analysis that we did, we assume $f$ was \emph{bounded} on the region
$\mathcal D$.  We could still do order analysis on an unbounded function, but it would take more 
care\footnote{
	Order analysis on $\sum_{i=1}^n i$ shows that it is order $n^2$, not order $n$.
	This is because the part inside the sum, $i$, is not bounded.
}.  Further, it'd be nice if there were a quicker way to come up with an iterated
integral in various coordinate systems, rather than lugging around Riemann sums
all the time---and there is.

\subsection{The Volume Form}

The \emph{volume form} is a formal way to rewrite integrals as iterated integrals
with respect to different coordinate systems.  What is really does is give a way
to relate areas or volumes described in a particular coordinate system to ``true''
areas and volumes in Euclidean space---and if your mind is tingling with thoughts
of parameterizations, you're on the right track.


Let's think back to line integrals.  Suppose $\vec p:[a,b]\to\R^n$ is an
arc-length parameterization
of the curve $\mathcal S$. Then, if $f:\R^n\to\R$,
\[
	\int_{\mathcal S} f = \int_{a}^b f\circ \vec p(t) \d t.
\]
However, if $\vec q:[c,d]\to\R^n$ is a non-arc-length parameterization of $\mathcal S$, we have to
use a different formula.  Namely,
\[
	\int_{\mathcal S} f = \int_{c}^d f\circ \vec q(t)\norm{\vec q\,'(t)} \d t.
\]
In this expression, $\norm{\vec q\,'(t)}$ compensates for the fact that as the parameter
$t$ marches on, the arc-length traced out by $\vec q$ is non-uniform.  The volume form does
the same thing but for parameterizations of multi-dimensional regions.


\begin{definition}
	Suppose $\mathcal F$ is a coordinate system for $\R^2$ with a relationship
	to rectangular coordinates given by $(x,y)=\vec f(a,b)$.
	The \emph{pre-volume form} associated with $\mathcal F$ at the point $(a,b)$
	is written $\Delta V(a,b)$ and is defined to be the area of
	$\vec f\big([a,a+\Delta a]\times [b,b+\Delta b]\big)$.
	
	The \emph{volume form} associated with $\mathcal F$ at the point $(a,b)$
	is the infinitesimal $\d V(a,b)=V(a,b)\d a\d b$ where $V:\R^2\to\R$ is the 
	unique function satisfying
	\[
		\lim_{\Delta a,\Delta b\to 0} \frac{\Delta V(a,b)}{V(a,b)\Delta a\Delta b}=1.
	\]
\end{definition}

There are a few peculiarities in the definition of the volume form.
First, the word ``infinitesimal'' shows up in the definition.  As of yet,
this has not been a formal term for us.  Further, the volume form has
``$\d a\d b$'' built into it.  This is because the term \emph{volume form}
comes from the field of differential geometry.  In differential geometry,
$\d a$ has a unique meaning all by itself and is called a \emph{differential form}.  For
us, $\d a$  by itself has no precise
meaning and only appears as a notational book-keeping device in integrals and derivatives.
Of course, the intuitive meaning of $\d a$ is that it is an infinitesimal quantity.
At this point, we won't dive into the details of how and why differential geometry and differential forms
work, but we
will use the language of differential geometry.

\begin{example}
	Compute the volume form for polar coordinates.

	We know that $p:[0,\infty)\times [0,2\pi)\to\R^2$ given 
	by $\vec p(r,\theta)=(r\cos\theta,r\sin\theta)$ converts from polar to rectangular
	coordinates.  Further, we already computed the pre-volume form for polar
	coordinates.  Namely, $\Delta V(r,\theta) = r\Delta r\Delta \theta+\tfrac{1}{2}(\Delta r)^2\Delta \theta$.

	XXX Figure


	To compute the volume form for polar coordinates, we need to find a function $V$ so that
	\begin{align*}
		\lim_{\Delta r,\Delta \theta\to 0}
		\frac{r\Delta r\Delta \theta+\tfrac{1}{2}(\Delta r)^2\Delta \theta}{V(r,\theta)\Delta r\Delta\theta}
		&=\lim_{\Delta r,\Delta \theta\to 0}
		\frac{r}{V(r,\theta)} + \frac{\Delta r}{2V(r,\theta)}\\
		&=\frac{r}{V(r,\theta)} = 1.
	\end{align*}
	There's a clear choice.  Namely, $V(r,\theta)=r$.  Therefore, the volume form for
	polar coordinates is $r\d r\d \theta$.

\end{example}

\begin{exercises}
\end{exercises}

\section{Surface Integrals}

\begin{exercises}
\end{exercises}
