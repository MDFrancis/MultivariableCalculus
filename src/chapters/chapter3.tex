\emph{Parameterization}\index{parameterization} is a mouthful, but the
fundamental idea of a parameterization is to describe one object in
terms of another.  For example, consider the line $\ell$ described
by the equation $y=2x$.  By its nature, $\ell$ is a set.
Using set-builder notation, we could write
\[
	\ell = \Set*{\mat{x\\y}\given y=2x}.
\]
But, we could also write $\ell$ in vector form as 
\[
	\mat{x\\y}=t\mat{1\\2}.
\]
Writing $\ell$ in vector form shows a pairing between scalars $t\in \R$
and points on $\ell$.  In many ways, $\ell$ is the same as $\R$, it's
just sitting in two-dimensional space instead of being on its own.

Taking a more technical viewpoint, we may consider $\ell$ to be the range of 
a vector-valued function.  Define $\vec p(t) = t\mat{1\\2}$.  Then,
\[
	\ell = \Range(\vec p) = \Set*{\vec x\given \vec x=\vec p(t)\text{ for some }t\in\R}.
\]
Now we have something special.  The function $\vec p:\R\to\R^2$ has domain $\R$ and outputs
every point on the line $\ell$ exactly once.  In other words, we've described
$\ell$ in terms of $\R$ and $\vec p$.  We could make a further assertion
that \emph{anything that you could learn by studying $\ell$, you could learn
by studying $\R$ and $\vec p$}.

However, there are other ways to create functions that describe $\ell$.
For example, consider $\vec q:\R\to\R^2$ where $\vec q(t)=2t\vec d$.  Again,
$\ell = \Range(\vec q)$ and so everything we could possibly learn about
$\ell$, we could learn by studying $\R$ and $\vec q$.
We call both $\vec p$ and $\vec q$ \emph{parameterizations of $\ell$ by $\R$}.

\begin{definition}[Parameterization]
	A \emph{parameterization} of an object $X$ by an object $Y$ is a continuous
	function $p:Y\to X$ with the added conditions that $p$ is one-to-one\footnote{
	Sometimes we will drop the requirement that a parameterization be one-to-one,
	but for now we'll be strict about it.}
	and $\Range(p)= Y$.  In this case $p$ is called a \emph{parameterization} and $Y$ is
	called the \emph{parameter}.
\end{definition}

This definition is fairly abstract, which will come in handy later, but for
now, we will think of $X$ as being some curve in $\R^n$ and $Y$ as being an
interval of real numbers.

\begin{example}[A Circle]
	Let $\mathcal C\subseteq \R^2$ be the unit circle centered at the origin.
	We can parameterize $\mathcal C$ by angles in $[0,2\pi)$.  Consider the function
	$\vec p:[0,2\pi)\to\mathcal C$ defined by
	\[
		\vec p(\theta) = \mat{\cos\theta \\\sin \theta}.	
	\]
	Here, $\vec p$ traces out $\mathcal C$ starting at the point $(1,0)$ and
	moving counter clockwise as the parameter $\theta$ increases.
\end{example}
\begin{example}[A Circle Again]
	Let $\mathcal C\subseteq \R^2$ be the unit circle centered at the origin.
	We will parameterize $\mathcal C$ by the interval $[0,1)$.  Here we might
	imagine that our parameter $t\in[0,1)$ represents a point that is $t$-percentage
	around the circle.  
	
	Recall $\vec p(\theta)=\mat{\cos \theta\\\sin\theta}$, which parameterizes
	$\mathcal C$ based on angles.  Now, consider the function $w(t) =2\pi t$.
	$w$ inputs numbers in $[0,1)$ and outputs angles in $[0,2\pi)$.  We should
	now be able to use $w$ to parameterize $\mathcal C$ in the desired way.
	After all, if we convert $[0,1)$ to $[0,2\pi)$ to $\mathcal C$, we win!

	Let the parameterization
		$\vec q:[0,1)\to\mathcal C$ be defined as $\vec q=\vec p\circ w$.
	Explicitly,
	\[
		\vec q(t) = \vec p\circ w(t) = \vec p(2\pi t) = \mat{\sin 2\pi t\\\cos 2\pi t}.	
	\]
\end{example}

\begin{exercise}
	Parameterize the unit circle $\mathcal C\subseteq \R^2$ by the interval $[1/2,1)$.
\end{exercise}
\begin{exercise}
	Let $\ell$ be the line segment connecting $(0,0)$ and $(1,1)$.
	Explain why $\vec p:[-1,1]\to\ell$ given by $p(t)=(t^2,t^2)$ is \emph{not}
	a parameterization.
\end{exercise}

\section{Speed and Velocity of a Parameterization}

In our day-to-day life, almost without thinking, we make a comparison between real numbers
and time.  Time has a forwards and backwards, which we equate to
the real number's increasing and decreasing.  We might even say we parameterize \emph{time}
by the real numbers.  Thus, if $\vec p:[a,b]\to \mathcal S$ is a parameterization 
of the curve $\mathcal S$ by the interval $[a,b]$, we could think of $\vec p$ as describing
the motion of a particle---at time $t\in[a,b]$ the particle is at $\vec p(t)$.

Interpreting parameterizations in this way, the \emph{speed}\index{speed} of a parameterization
should be the rate of change of distance with respect to time and the \emph{velocity}\index{velocity}
of a parameterization should be the rate of change of displacement with respect to time.

Suppose $\vec p:[a,b]\to\mathcal S$ is a parameterization of $\mathcal S$ and $t\in[a,b]$
represents time.  The \emph{displacement} of $\vec p$ from time $t$ to time $t+\Delta t$ is
$\vec p(t+\Delta t)-\vec p(t)$ and the change in \emph{distance} is $\norm{\vec p(t+\Delta t) - \vec p(t)}$.
Thus, if $\Delta t$ is small, the velocity at time $t$ can be approximated by
\[
		\Vel \vec p(t) \approx \frac{\vec p(t+\Delta t) - \vec p(t)}{\Delta t}	
\]
and the speed\footnote{
		Recall that speed is always positive; if a particle is moving with speed $2$
		and we then ran the particle back in time, it would still move at speed $2$,
		so speed is not $\text{distance}/\Delta t$, it is $\text{distance}/\abs{\Delta t}$.
		} by
\[
	\Speed \vec p(t) \approx \frac{\norm{\vec p(t+\Delta t) - \vec p(t)}}{\abs{\Delta t}}.	
\]
Taking limits, we arrive at exact rates of change, which leads us to the following definitions.
\begin{definition}[Speed]
	Let $\vec p:[a,b]\to\mathcal S$ be a parameterization of $\mathcal S$.  The
	\emph{speed} of $\vec p$ at the time $t\in[a,b]$ is 
	\[
		\Speed \vec p(t) = \lim_{\Delta t\to 0} \frac{\norm{\vec p(t+\Delta t) - \vec p(t)}}{\abs{\Delta t}}.	
	\]
\end{definition}
\begin{definition}[Velocity]
	Let $\vec p:[a,b]\to\mathcal S$ be a parameterization of $\mathcal S$.  The
	\emph{velocity} of $\vec p$ at the time $t\in[a,b]$ is 
	\[
		\Vel \vec p(t) = \lim_{\Delta t\to 0} \frac{\vec p(t+\Delta t) - \vec p(t)}{\Delta t}.	
	\]
\end{definition}

Both the definition of speed and the definition of velocity look a lot like the definition
of the derivative.  In fact, if $\vec p$ were a scalar valued function, the velocity of $\vec p$
would be exactly the derivative of $\vec p$.  For this reason, we will define a notation
similar to that of the derivative you're familiar with.  From now on, the following notations mean
the same thing:
\[
	\Vel \vec p(t) = \vec p\,'(t) = \frac{\d}{\d t}\vec p(t) = \frac{\d \vec p}{\d t}(t).
\]

Let's try to use our new definition.  Let $\vec r(t) = \mat{\cos t\\\sin t}$.  Now,
\begin{align*}
	\Vel \vec r(t) &=\lim_{\Delta t\to 0} \frac{\mat{\cos(t+\Delta t)\\\sin(t+\Delta t)} - \mat{\cos t\\\sin t}}{\Delta t}\\
	&=\lim_{\Delta t\to 0} \mat{
		\displaystyle\frac{\cos(t+\Delta t)-\cos t}{\Delta t} \\\displaystyle\frac{\sin(t+\Delta t)-\sin t}{\Delta t}
		}.
\end{align*}
At this point we should pause.  We don't know how to take limits of vectors.  Fortunately the rule
is simple enough---to take a limit of a vector, take the limit of each of its components\footnote{
	As intuitive as it sounds, this rule actually has a proof which relies on the definition
	of limit and the continuity of $\norm{\:\cdot\:}$.
}.  Thus we see
\begin{align*}
	\Vel \vec r(t) 
	&=\lim_{\Delta t\to 0} \mat{
		\displaystyle\frac{\cos(t+\Delta t)-\cos t}{\Delta t} \\\frac{\displaystyle\sin(t+\Delta t)-\sin t}{\Delta t}}\\
	&=\mat{
		\displaystyle\lim_{\Delta t\to 0} \frac{\cos(t+\Delta t)-\cos t}{\Delta t} \\\displaystyle\lim_{\Delta t\to 0} \frac{\sin(t+\Delta t)-\sin t}{\Delta t}
		}\\
	&=\mat{\cos'(t)\\\sin'(t)} = \mat{-\sin t\\\cos t}.
\end{align*}
Our use of the notation $\vec r\,'(t)$ for $\Vel \vec r(t)$ seems further justified.

Speed also appears to be a derivative.  From physics we know speed is the magnitude of 
velocity.  Mathematically, we can prove it.
\begin{theorem}
	For a parameterization $\vec p:\R\to\R^n$ where $\Vel \vec p(t)$
	exists, we have
	\[
		\Speed \vec p(t) = \norm{\Vel \vec p(t)} = \norm{\vec p\,'(t)}.
	\]
\end{theorem}
\begin{proof}
	The proof relies on the continuity of $\norm{\:\cdot\:}$.  Since $\norm{\:\cdot\:}$
	is continuous, we may freely move limits in and out.  Thus
	\begin{align*}
		\Speed \vec p(t) &= \lim_{\Delta t\to 0} \frac{\norm{\vec p(t+\Delta t) - \vec p(t)}}{\abs{\Delta t}}\\
		&= \lim_{\Delta t\to 0} \norm*{\frac{\vec p(t+\Delta t) - \vec p(t)}{\Delta t}}\\
		&=\norm*{\lim_{\Delta t\to 0} \frac{\vec p(t+\Delta t) - \vec p(t)}{\Delta t}}
		=\norm{\Vel \vec p(t)}.
	\end{align*}
\end{proof}

\subsection{Arc-length}
Let $\mathcal S\subseteq \R^n$ be a curve parameterized by $\vec p:[a,b]\to\R^n$.  
The \emph{arc-length}\index{arc length} of $\mathcal S$
should be the length of $\mathcal S$ if you somehow untwisted $\mathcal S$ into a straight
line without stretching anything.  One of the big ideas of calculus is that 
we can handle curvy things by chopping them up into little pieces, computing
for each piece, and then adding them back together.  We use the same principle 
to define arc-length.

In essence, we will divide our curve $\mathcal S$ into many tiny line segments,
add up the lengths of those line segments and take a limit as our line segments
get tinier.  A parameterization provides us with a way to do this.  Since parameterizations
are continuous, if we chop the domain of the parameterization into tiny pieces,
we will have chopped the range into tiny pieces.

XXX Figure

To find the arc-length of a curve, we will approximate each tiny piece with a straight
line segment connecting the endpoints.  We then add up the lengths of all tiny segments
and take a limit as our segment's length goes to zero.

\begin{definition}[Arc-length]
	Let $\mathcal S\subseteq \R^n$ be a curve parameterized by $\vec p:[a,b]\to\mathcal S$. 
	The \emph{arc-length}
	of $\mathcal S$ is 
	\[
		\Arclen\mathcal S = \lim_{\Delta t\to 0^+} \sum_{i=1}^{\frac{b-a}{\Delta t}} 
		\,\norm{\vec p(a+(i-1)\Delta t)-\vec p(a+i\Delta t)}.
	\]
\end{definition}

There's something unsatisfying about this definition, though.  We used a parameterization
of $\mathcal S$ to compute the arc-length of $\mathcal S$.  But $\mathcal S\subseteq \R^n$ 
is a curve regardless of whether or not it has a parameterization, and if you use a different
parameterization, you should get the same arc length for $\mathcal S$.  If you're worried about
this, good!  You're thinking carefully!  We won't show it here, but in fact no matter what parameterization
you use for a curve, this definition will always produce the same arc length.

There's another reason we might be unhappy with this definition.  Limits of sums are
hard to compute!  However, the sum involved in arc-length looks very close to a Riemann sum.
If we can rewrite it exactly as a Riemann sum, we can replace it with an integral.  With some
superficial manipulation we see

\begin{align*}
	\Arclen\mathcal S 
	&= \lim_{\Delta t\to 0^+} \sum_{i=1}^{\frac{b-a}{\Delta t}} 
	\,\norm{\vec p(a+(i-1)\Delta t)-\vec p(a+i\Delta t)}\\
	&= \lim_{\Delta t\to 0^+} \sum_{i=1}^{\frac{b-a}{\Delta t}} 
	\,\frac{\norm{\vec p(a+(i-1)\Delta t)-\vec p(a+i\Delta t)}}{\Delta t}\Delta t\\
	&=\int_{a}^b \Speed \vec p(t)\,\d t.
\end{align*}

For the last equality, we noticed that 
$\lim_{\Delta t\to 0^+} \frac{\norm{\vec p(t)-\vec p(t+\Delta t)}}{\Delta t}=\Speed \vec p(t)$,
which involved switching a limit and an infinite sum.  In order to do this rigorously, we need
a mathematical proof that it is logically valid.  Such a proof is the subject of a course in
\emph{real analysis}, and won't be covered here, but it's always good to keep track of what
you've actually proved and what you've been told is true\footnote{ In order to prove that swapping
the limit and sum is valid, we actually need extra assumptions on $\vec p$.  If we make $\vec p$
\emph{differentiable} rather than merely \emph{continuous}, we can prove that the swap is valid.}.

Speed is easy to calculate, and we have a better handle on calculating integrals than we do 
limits of sums, so now we have a chance of calculating arc-length.

\begin{example}
	\label{EXHARDARCLEN}
We shall find the length of the parabola with equation $y = x^2$
on the interval $-1\le x \le 1$.   A parametric representation
	of the parabola is $\vec p(t) = (t,t^2)$
where $-1 \le t \le 1$.  
Now,
\[
	\frac{\d\vec p}{\d t} = \mat{1\\2t},
\]
so $\Speed \vec p(t)=\norm{\vec p\,'(t)} = \sqrt{ 1 + 4t^2}$.
Hence,
\[
  L = \int_{-1}^1 \sqrt{1 + 4t^2} dt = \sqrt 5 + 
\frac 14\ln\frac{\sqrt 5 + 2}{\sqrt 5 - 2}.
\]
\end{example}

As you can see from Example \ref{EXHARDARCLEN}, the integrals involved in computing
arc length can be difficult.  In fact, most of them don't have an elementary form, which
means in the real world we often approximate arc length directly from the Riemann sum
rather than calculate it exactly.

\begin{exercises}
\end{exercises}

\section{Arc-length Parameterization}

Recall a parameterization is a relation between two objects.  If a curve $\mathcal S\subseteq\R^n$
is parameterized by $\R$, it means there is a continuous, one-to-one function $\vec p:\R\to\mathcal S$.
This function can be thought of as a map from $\R$ to $\mathcal S$.  Any interval $[a,b]\subseteq\R$
corresponds to a segment $\vec p([a,b])\subseteq \mathcal S$.

XXX Figure

Alternatively, we may think of $\vec p:\R\to\mathcal S$ as a function that picks up the real line,
stretches, twists, and warps it, and sticks it into $\R^n$ in the shape of $\mathcal S$.
In this sense, not all parameterizations are created equally.  Some significantly stretch and warp
and others barely do at all.  The least stretchy type of parameterization is called an \emph{arc-length
parameterization}\index{arc-length parameterization}.

Before we define arc-length parameterization, let's introduce some notation.  If $\mathcal S\subseteq \R^n$
is a curve parameterized by $\vec p:\R\to\mathcal S$, then
\[
	\Arclenfrom{\vec p}{a}{b}  = \text{ arc length of $\mathcal S$ between $\vec p(a)$ and $\vec p(b)$}.
\]
We might read $\Arclenfrom{\vec p}{a}{b}$ as the ``arc length of the curve traced by $\vec p(t)$ from
$t=a$ to $t=b$.''  Using notion for the \emph{image} of a set, we can also write
\[
	\Arclenfrom{\vec p}{a}{b} = \Arclen \vec p([a,b]).
\]

\begin{definition}
	Let $\mathcal S\subseteq\R^n$ be a curve and $\vec p:\R\to\mathcal S$ be a parameterization.
	The parameterization $\vec p$ is called an \emph{arc-length parameterization} if for $b\geq a$,
	\[
		\Arclenfrom{\vec p}{a}{b}=b-a.
	\]
\end{definition}

In plain language, if $\vec p$ is an arc-length parameterization, then the distance traveled in parameter
space is the same as the distance traveled along the curve.

XXX Figure

\begin{example}
	Find an arc-length parameterization of the line $\ell$ which passes through
	the origin and has a direction vector $\vec d=\xhat+2\yhat+\zhat$.

	Let's name our parameterization $\vec p$ and see if we can build up
	a formula for $\vec p$.
	Since $\ell$ passes through the origin, let's have $\vec p(0)=\vec 0$.
	Now, $\vec p(1)$ has to be a point on $\ell$ that is distance $1$ from $\vec 0$.
	There are two such points, so we'll arbitrarily declare
	$
		\vec p(1) = \frac{1}{\sqrt{6}}(1,2,1).
	$
	The point $\vec p(2)$ must be distance $1$ from $\vec p(1)$ and distance $2$ from $\vec p(0)$.
	Since parameterizations aren't allowed to double-back, we only have one choice.  Namely,
	$
		\vec p(2) = \frac{2}{\sqrt{6}}(1,2,1).
	$
	Noticing the patter, let's guess
	\[
		\vec p(t) = \frac{t}{\sqrt{6}}\mat{1\\2\\1}.
	\]
	Now we'll verify that $\vec p$ is an arc-length parameterization.  Since $\ell$ is
	a straight line this is easy:
	\[
		\Arclenfrom{\vec p}{a}{b}=\norm{\vec p(b)-\vec p(a)} = \abs*{\tfrac{b-a}{\sqrt{6}}}\norm*{\mat{1\\2\\1}}
		=\abs{b-a},
	\]
	and so $\vec p$ is an arc-length parameterization.
\end{example}

Arc-length parameterizations can also be characterized by their speed.  For an arc-length
parameterization, the distance traveled in parameter space must be equal to the distance
traveled along the curve.  Therefore, an arc-length parameterization must also move at
speed 1.

\begin{theorem}
	Let $\mathcal S\subseteq\R^n$ be a curve and $\vec p:\R\to\mathcal S$
	a parameterization.  The parameterization $\vec p$ is an arc-length
	parameterization if and only if $\Speed\vec p=1$.
\end{theorem}
\begin{proof}
	Suppose $\vec p$ is a parameterization of $\mathcal S$ and $\Speed \vec p=1$.  Then
	\[
		\Arclenfrom{\vec p}{a}{b} = \int_a^b\Speed\vec p(t)\,\d t = \int_a^b1\d t = b-a.
	\]

	Now, suppose that $\vec p$ satisfies $\Arclenfrom{\vec p}{a}{b}=b-a$.  We then know if
	$\Delta t>0$ is small
	\[
		\frac{\norm{\vec p(t+\Delta t)-\vec p(t)}}{\Delta t} \approx \frac{\Arclenfrom{\vec p}{t}{t+\Delta t}}{\Delta t}
		= \frac{(t+\Delta t) - t}{\Delta t} = 1.
	\]
	After taking a limit as $\Delta t\to 0$, ``$\approx$'' will turn into ``$=$''.

	To make this argument completely rigorous, we will have to do a little bit of
	mathematical gymnastics.  Since the shortest distance between two points is a straight
	line, what we really know is
	\[
		\norm{\vec p(t+\Delta t)-\vec p(t)} \leq \Arclenfrom{\vec p}{t}{t+\Delta t}
		 = \Delta t
	\]
	for all $\Delta t$.  Now, taking a limit as $\Delta t\to 0$, we deduce $\Speed \vec p(t)\leq 1$.  But,
	\[
		b-a = \Arclenfrom{\vec p}{a}{b} = \int_a^b \Speed \vec p(t)\,\d t \leq
		\int_a^b1\d t  = b-a,
	\]
	and so if $\Speed \vec p$ is a continuous function, $\Speed\vec p(t)=1$ for all $t$.
	Even if speed is not continuous, we can argue that $\Speed \vec p$ is \emph{essentially} 1.\footnote{
	Here, the word \emph{essentially} is a technical term coming from real analysis.
		}
\end{proof}

Now we have a new way of thinking about arc-length parameterizations and a new way of creating them.
Instead of working with arc length directly, we can attempt to manipulate \emph{speed}.

\begin{example}
	\label{EXARCLENPARAM}
	Find an arc-length parameterization of $\mathcal C$, the circle of radius $2$ centered at the origin.

	We already know a parameterization of $\mathcal C$.  Namely,
	\[
		\vec r(t) =\mat{2\cos t\\2\sin t}.
	\]
	However, $\Speed \vec r(t) =\norm{\vec r\,'(t)} = \sqrt{(-2\sin t)^2+(2\cos t)^2} = 2$.  If we could
	somehow \emph{slow down} time, we could slow down the speed to be $1$, giving an arc-length parameterization.

	Let $w(t)=t/2$.  The function $w$ stretches time by a factor of $2$.  Define
	\[
		\vec p(t) = \vec r\circ w(t) = \mat{2\cos t/2\\2\sin t/2}.
	\]
	Now, 
	\begin{align*}
		\Speed \vec p(t) = \norm{\vec p\,'(t)}&=\norm{(\vec r\circ w)'(t)} = \norm{(\vec r\,'\circ w(t))w'(t)}\\
		&=\abs{w'(t)}\norm{\vec r\,'\circ w(t)} = \tfrac{1}{2}2=1.
	\end{align*}

	In this computation we made judicious use of the chain rule, however we could have computed
	directly from our formula for $\vec p$.  Now,
	since $\Speed\vec p(t)=1$, the function $\vec p$ is an arc-length parameterization.
\end{example}

\begin{exercise}
	\label{EXCANTMUL}
	Let $\mathcal C$ and $\vec r$ be as in Example \ref{EXARCLENPARAM}.  The function $\vec q(t)=\tfrac{1}{2}\vec r(t)$
	has speed 1 and so is an arc-length parameterization of something.  Explain why $\vec q$ is \emph{not}
	an arc-length parameterization of $\mathcal C$.
\end{exercise}

In Example \ref{EXARCLENPARAM}, we adjusted the speed of a parameterization by warping time.  Suppose
$\mathcal S\subseteq \R^n$ is a curve parameterized by $\vec p:\R\to \mathcal S$ and let $w:\R\to\R$ be 
a parameterization of $\R$.  In other words, $w$ stretches or squishes (or flips) $\R$ by varying amounts.
Now consider
\[
	\vec r=\vec p\circ w.
\]
The function $\vec r$ has domain $\R$ and range $\mathcal S$.  Further, since $\vec p$ and $w$ are both
one-to-one and continuous, we know $\vec r$ is one-to-one and continuous.  Thus, $\vec r$ is a parameterization
of $\mathcal S$.  And, by using the chain rule,
\[
	\vec r\,'(t) = (\vec p\circ w)'(t) = w'(t)\big[\vec p\,'\circ \vec w(t)\big],
\]
so
\[
	\Speed\vec r\text{ at time $t$} = \abs{w'(t)}\big[\Speed \vec p\text{ at time $w(t)$}\big].
\]
It is important to note that we must compose $\vec p$ and $w$ in order to get a parameterization of $\mathcal S$.
See Exercise \ref{EXCANTMUL} for an example of why you cannot multiply $\vec p$ and $w$.

With the idea of stretching time in the back of our minds, let's work through a hypothetical example.

Suppose $\mathcal S\subseteq \R^n$ is a curve parameterized by $\vec p:\R\to\mathcal S$ and we've computed
the following table of values.
\begin{center}
	\begin{tabular}{c|c}
		$t$ & $\Arclenfrom{\vec p}{0}{t}$\\
		\hline
		$0$ & $0$\\
		$1$ & $2$\\
		$2$ & $3.5$\\
		$3$ & $6$\\
		$4$ & $7$
	\end{tabular}
\end{center}

XXX Figure

We'd like to find a time-stretching function $w:\R\to\R$ so that $\vec r=\vec p\circ w$ is an arc-length
parameterization of $\mathcal S$.  In other words, we need
\[
	\Arclenfrom{\vec r}{0}{t} = \Arclenfrom{(\vec p\circ w)}{0}{t}=t.
\]
In particular, $\Arclenfrom{\vec r}{0}{0}=0$ (this we get for free) and $\Arclenfrom{\vec r}{0}{2}=2$.
From the table, we know that the arc length from $\vec p(0)$ to $\vec p(1)$ is $2$, and so we need
$w(2) =1$.  This way $\vec r(2) = \vec p(w (2)) = \vec p(1)$.  Continuing in this way, we get the following
table of values for $w$.
\begin{center}
	\begin{tabular}{c|c}
		$t$ & $w(t)$\\
		\hline
		$0$ & $0$\\
		$2$ & $1$\\
		$3.5$ & $2$\\
		$6$ & $3$\\
		$7$ & $4$
	\end{tabular}
\end{center}

The function $w$ is just the inverse of the function $\Arclenfrom{\vec p}{0}{t}$!  This also makes sense
from a purely algebraic perspective.  Consider
\[
	x=\Arclenfrom{\vec r}{0}{x} = \Arclenfrom{(\vec p\circ w)}{0}{x}=\Arclenfrom{(\vec p)}{0}{w(x)}.
\]
Replacing $x$ with $w^{-1}(t)$, we see
\[
	w^{-1}(t) = \Arclenfrom{\vec p}{0}{w\circ w^{-1}(t)} = \Arclenfrom{\vec p}{0}{t}.
\]
This means the inverse of $w$ is $\Arclenfrom{\vec p}{0}{t}$ and so the inverse of $\Arclenfrom{\vec p}{0}{t}$
must be $w$.\footnote{ Recall that for an invertible function $f$, we have $(f^{-1})^{-1}=f$.}

We now have a concrete way to find an arc-length parameterization.
\begin{example}
	Let $\mathcal R$ be the ray parameterized by 
	$\vec p:[0,\infty)\to\R^2$ where $\vec p(t) = (t^2,2t^2)$.  Find an arc-length
	parameterization of $\mathcal R$.

	We'll start by finding the arc-length function for $\vec p$.
	\begin{align*}
		a(t)=\Arclenfrom{\vec p}{0}{t} &= \int_0^t \norm{\vec p\,'(x)}\d x\\
		&=\int_0^t \sqrt{(2x)^2+(4x)^2}\d x = t^2\sqrt{5}.
	\end{align*}
	Now define $w(t)=a^{-1}(t)$.  Since $t\geq 0$, the inverse of $a$ is well defined
	and is given by
	\[
		w(t)=a^{-1}(t) = \sqrt{t/\sqrt{5}}.
	\]
	Now, define $\vec r(t) = \vec p\circ w(t)=(t/\sqrt{5},2t/\sqrt{5})$.  The parameterization
	$\vec r$ is an arc-length parameterization!

	Of course we didn't need to go through all this work in this case.  If all we wanted was
	to find an arc-length parameterization of $\mathcal R$ we could create one directly using
	geometry, since we know how to parameterize lines and rays.
\end{example}

\subsection{Explicit Arc-length Parameterizations}
The idea of arc-length parameterization is very important.  However, for most curves,
we're hopeless in finding a formula for the arc-length parameterization.  We can for a hand
full of curves, like a circle, a line, a helix, but even something as simple as an ellipse
cannot be arc-length parameterized with elementary functions.

Why is it so hard?  Well, integrals in general are hard.  Most formulas cannot be integrated in closed
form.  Integrals involving square roots are even harder to evaluate.  And, even if you manage
to integrate to find the arc-length function, you still have to invert that function\footnote{ If you don't
believe me, go ahead and 
try to find the inverse of the function $f(x)=xe^x$. }.  So, if finding formulas for arc-length parameterizations
is so hard, why do we bother with them at all?  The answer is that the \emph{idea} of an arc-length parameterization
is incredibly useful. Its mere existence will aid our thinking.  And, in many of the problems we will be solving,
the arc-length parameterization will somehow get cancelled out and we won't ever need to find a formula for it.

\begin{exercises}
\end{exercises}

\section{Acceleration and Curvature}
In the Newtonian mechanics of one-dimensional motion, acceleration\index{acceleration} is the second
derivative of position with respect to time.  In $\R^n$ we define it in the same way.

\begin{definition}[Acceleration]
	Let $\vec p:\R\to\mathcal S$ be a parameterization of $\mathcal S$.  The \emph{acceleration}
	of $\vec p$ is 
	\[
		\Accel \vec p(t) = (\Vel \vec p)'(t)=\vec p\,''(t).
	\]
\end{definition}

Just like velocity, acceleration is a \emph{vector}.  Let's consider two examples.  Define
\[
	\vec l(t) = \mat{t^2\\t^2}\qquad\text{and}\qquad\vec c(t) = \mat{\cos t\\\sin t}.
\]
Here $\vec l$ parameterizes a ray and $\vec c$ a circle.  Further, $\vec l$ traces along the ray
faster and faster, whereas $\vec c$ has constant speed as it traces the circle.
We compute 
\[
	\Accel \vec l(t) = \mat{2\\2}\qquad\text{and}\qquad \Accel\vec c(t) = \mat{-\cos t\\ -\sin t},
\]
and see that the acceleration of $\vec l$ is constant whereas the acceleration of $\vec c$ is not.  Further,
$\norm{\Accel \vec l(t)}=2$ and $\norm{\Accel \vec c(t)}=1$, and so the magnitude of the acceleration of
both $\vec l$ and $\vec c$ is constant.

In the past, you might have distinguished linear acceleration (running faster and faster along a straight line) from
centripetal acceleration (the acceleration you experience by moving at a constant speed
around a circle).  With vectors, these two types of acceleration are unified into a single vector.  

In the previous example, $\vec l$ had purely linear acceleration and
the acceleration vector pointed tangent to the path $\vec l$ traced.
Analyzing $\vec c$, we see $\vec c$ had purely centripetal acceleration and the acceleration was orthogonal
to the curve it traced.
What happens if we mix the two types of acceleration?

Consider
\[
	\vec r(t)=\mat{\cos t^2\\\sin t^2}.
\]
Computing, 
\[
	\Accel \vec r(t) = \mat{-2(\sin t^2+2t^2\cos t^2)\\ 2(\cos t^2-2t^2\sin t^2)}.
\]
There is no clear relationship between the curve $\vec r$ traces and the acceleration vector
for $\vec r$.  To see this relationship, we need to decompose $\Accel \vec r$ into its tangential
and normal components\index{normal acceleration}\index{tangential acceleration}.

\begin{definition}[Tangential and Normal Acceleration]
	Let $\vec p:\R\to\mathcal S$ be a parameterization of $\mathcal S$.  Then
	$\Accel \vec p$ can be written as
	\[
		\Accel \vec p(t) = \vec a_T(t)+\vec a_N(t)
	\]
	where $\vec a_T(t)$ is tangent to $\mathcal S$ at the point $\vec p(t)$ and $\vec a_N(t)$
	is orthogonal to $\mathcal S$ at the point $\vec p(t)$.  In this case, $\vec a_T(t)$ is
	called the \emph{tangential component of the acceleration} and $\vec a_N(t)$ is
	called the \emph{normal component of the acceleration} of $\vec p$.
\end{definition}

\begin{example}
	Let $\vec r(t)=(\cos t^2,\sin t^2)$.  Find the tangential and normal components of
	the acceleration of $\vec r$.

	Earlier we computed
	\[
		\Accel \vec r(t) = \mat{-2(\sin t^2+2t^2\cos t^2)\\ 2(\cos t^2-2t^2\sin t^2)}.
	\]
	We can use projections to split $\Accel \vec r$ into its tangential and normal components.

	Recall $\vec r\,'(t)$ is tangent to the curve $\vec r$ traces at the point $\vec r(t)$.  Thus,
	\[
		\vec a_T = \Proj_{\vec r\,'(t)} \Accel \vec r(t) = \mat{-2\sin t^2\\ 2\cos t^2},
	\]
	and 
	\[
		\vec a_N = \Accel\vec r(t) - \vec a_T(t) = \mat{-4t^2\cos t^2\\ -4t^2\sin t^2}.
	\]
\end{example}

Suppose now that $\vec r:\R\to\mathcal S$ is an arc-length parameterization of $\mathcal S$.  Since
the speed of $\vec r$ is constant, it is intuitive that the tangential component of 
the acceleration of $\vec r$ is zero.  Equipped with our knowledge of vectors, it won't be
so hard to prove our intuition.  But, it will be helpful to establish the product rule
for dot products.

\begin{exercise}
	Let $\vec a(t)=(a_x(t),a_y(t),a_z(t))$ and $\vec b(t)=(b_x(t),b_y(t),b_z(t))$ be parameterizations.
	Establish the \emph{product rule for dot products}\index{product rule for dot products}.  That is,
	show that
	\[
		\Big[\vec a(t)\cdot \vec b(t)\Big]' = \vec a\,'(t)\cdot \vec b(t)+\vec a(t)\cdot \vec b\,'(t).
	\]
\end{exercise}

\begin{theorem}
	\label{THMARCLENACCEL}
	If $\vec p:\R\to\mathcal S$ is an arc-length parameterization of $\mathcal S$ then
	$\Accel\vec p$ is always orthogonal to $\mathcal S$.  Equivalently,
	\[
		\vec p\,''(t)\cdot \vec p\,'(t)=0.
	\]
\end{theorem}
\begin{proof}
	Since $\vec p$ is an arc-length parameterization, we know 
	\[
		\sqrt{\vec p\,'(t)\cdot \vec p\,'(t)} = \norm{\vec p\,'(t)}=1.
	\]
	Squaring both sides we get the relationship
	\begin{equation}
		\label{EQARCLENACCEL}
		\vec p\,'(t)\cdot \vec p\,'(t)=1.
	\end{equation}
	Now we may take the derivative of both sides of Equation \eqref{EQARCLENACCEL} and apply the
	product rule for dot products to find
	\[
		0=\big[\vec p\,'\cdot \vec p\,'\big]'(t) = \vec p\,''(t)\cdot \vec p\,'(t)+
		\vec p\,'(t)\cdot \vec p\,''(t) = 2\vec p\,''(t)\cdot \vec p\,'(t),
	\]
	and so $\vec p\,''(t)$ and $\vec p\,'(t)$ are orthogonal.
\end{proof}

If you examine the proof of theorem \ref{THMARCLENACCEL} closely, you'll notice that we didn't
actually need the speed of our parameterization to be $1$.  The proof works just as well if
the speed is some other constant.

\subsection{Curvature}
For any curve $\mathcal S$ there are infinitely many choices of parameterizations, but in some
sense, there is only one arc-length parameterization.  An arc-length parameterization of
$\mathcal S$ is uniquely determined by a direction (forwards or backwards along $\mathcal S$)
and a starting position.  Thus, we might think of an arc-length parameterization as 
\emph{intrinsic} to a curve.

The \emph{curvature}\index{curvature} of a curve is a measure of how sharply a curve bends or
twists.  Curvature is another property of a curve---you don't need a parameterization to define
curvature---but it is a lot easier to define with reference to an arc-length parameterization.

\begin{definition}[Curvature]
	Let $\mathcal S\subseteq\R^n$ be a curve and let $\vec p:\R\to\mathcal S$ be an arc-length
	parameterization of $\mathcal S$.  The \emph{curvature} of $\mathcal S$ at the point
	$\vec p(t)$ is 
	\[
		\norm{\Accel \vec p(t)} = \norm{\vec p\,''(t)}.
	\]
\end{definition}

This definition of curvature can be made intuitive.  If $\vec p:\R\to\mathcal S$ is an arc-length parameterization,
all velocity vectors are unit length.  Therefore, all acceleration of $\vec p$ must come from
the velocity vectors changing direction (and not changing length).  If a curve has a sharp bend (high curvature),
the velocity vectors with rapidly change direction.  If a curve is generally flat (low curvature),
the velocity vectors hardly change direction at all.

XXX Figure

Curvature can be hard to calculate exactly, but it isn't so hard to eyeball.

\begin{example}
	Estimate the curvature at various points of the parabola $y=x^2$.

	XXX Finish and include numerics
\end{example}

\begin{exercises}
\end{exercises}

\section{Line Integrals}

\section{Multi-dimensional Parameterizations}
