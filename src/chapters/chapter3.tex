\emph{Parameterization}\index{parameterization} is a mouthful, but the
fundamental idea of a parameterization is to describe one object in
terms of another.  For example, consider the line $\ell$ described
by the equation $y=2x$.  By its nature, $\ell$ is a set.
Using set-builder notation, we could write
\[
	\ell = \Set*{\mat{x\\y}\given y=2x}.
\]
But, we could also write $\ell$ in vector form as 
\[
	\mat{x\\y}=t\mat{1\\2}.
\]
Writing $\ell$ in vector form shows a pairing between scalars $t\in \R$
and points on $\ell$.  In many ways, $\ell$ is the same as $\R$---it's
just sitting in two-dimensional space instead of being on its own.

Taking a more technical viewpoint, we may consider $\ell$ to be the range of 
a vector-valued function.  Define $\vec p(t) = t\mat{1\\2}$.  Then,
\[
	\ell = \Range(\vec p) = \Set*{\vec x\given \vec x=\vec p(t)\text{ for some }t\in\R}.
\]
Now we have something special.  The function $\vec p:\R\to\R^2$ has domain $\R$ and outputs
every point on the line $\ell$ exactly once.  In other words, we've described
$\ell$ in terms of $\R$ and $\vec p$.  We could make a further assertion
that \emph{anything that you could learn by studying $\ell$, you could learn
by studying $\R$ and $\vec p$}.

However, there are other ways to create functions that describe $\ell$.
For example, consider $\vec q:\R\to\R^2$ where $\vec q(t)=2t\vec d$.  Again,
$\ell = \Range(\vec q)$ and so everything we could possibly learn about
$\ell$, we could learn by studying $\R$ and $\vec q$.
We call both $\vec p$ and $\vec q$ \emph{parameterizations of $\ell$ by $\R$}.

\begin{definition}[Parameterization]
	A \emph{parameterization} of an object $X$ by an object $Y$ is a continuous
	function $p:Y\to X$ with the added conditions that $p$ is one-to-one\footnote{
	Sometimes we will drop the requirement that a parameterization be one-to-one,
	but for now we'll be strict about it.}
	and $\Range(p)= Y$.  In this case $p$ is called a \emph{parameterization} and $Y$ is
	called the \emph{parameter}.
\end{definition}

This definition is fairly abstract, which will come in handy later. For
now, we will think of $X$ as being some curve in $\R^n$ and $Y$ as being an
interval of real numbers.

\begin{example}[A Circle]
	Let $\mathcal C\subseteq \R^2$ be the unit circle centered at the origin.
	We can parameterize $\mathcal C$ by angles in $[0,2\pi)$.  Consider the function
	$\vec p:[0,2\pi)\to\mathcal C$ defined by
	\[
		\vec p(\theta) = \mat{\cos\theta \\\sin \theta}.	
	\]
	Here, $\vec p$ traces out $\mathcal C$ starting at the point $(1,0)$ and
	moving counter clockwise as the parameter $\theta$ increases.
\end{example}
\begin{example}[A Circle Again]
	Let $\mathcal C\subseteq \R^2$ be the unit circle centered at the origin.
	We will parameterize $\mathcal C$ by the interval $[0,1)$.  Here we might
	imagine that our parameter $t\in[0,1)$ represents a point that is $t$-percentage
	around the circle.  
	
	Recall $\vec p(\theta)=\mat{\cos \theta\\\sin\theta}$, which parameterizes
	$\mathcal C$ based on angles.
	
	Now, consider the function $w(t) =2\pi t$.
	$w$ inputs numbers in $[0,1)$ and outputs angles in $[0,2\pi)$.  We should
	now be able to use $w$ to parameterize $\mathcal C$ in the desired way.
	After all, if we convert $[0,1)$ to $[0,2\pi)$ to $\mathcal C$, we win!

	Let the parameterization
		$\vec q:[0,1)\to\mathcal C$ be defined as $\vec q=\vec p\circ w$.
	Explicitly,
	\[
		\vec q(t) = \vec p\circ w(t) = \vec p(2\pi t) = \mat{\sin 2\pi t\\\cos 2\pi t}.	
	\]
\end{example}

\begin{exercise}
	Parameterize the unit circle $\mathcal C\subseteq \R^2$ by the interval $[1/2,1)$.
\end{exercise}
\begin{exercise}
	Let $\ell$ be the line segment connecting $(0,0)$ and $(1,1)$.
	Explain why $\vec p:[-1,1]\to\ell$ given by $p(t)=(t^2,t^2)$ is \emph{not}
	a parameterization.
\end{exercise}

\section{Speed and Velocity of a Parameterization}

In our day-to-day life, almost without thinking, we make a comparison between real numbers
and time.  Time has a forwards and backwards, which we equate to
the real number's increasing and decreasing.  We might even say we parameterize \emph{time}
by the real numbers.  Thus, if $\vec p:[a,b]\to \mathcal S$ is a parameterization 
of the curve $\mathcal S$ by the interval $[a,b]$, we could think of $\vec p$ as describing
the motion of a particle---at time $t\in[a,b]$ the particle is at $\vec p(t)$.

Interpreting parameterizations in this way, the \emph{speed}\index{speed} of a parameterization
should be the rate of change of distance with respect to time and the \emph{velocity}\index{velocity}
of a parameterization should be the rate of change of displacement with respect to time.

Suppose $\vec p:[a,b]\to\mathcal S$ is a parameterization of $\mathcal S$ and $t\in[a,b]$
represents time.  The \emph{displacement} of $\vec p$ from time $t$ to time $t+\Delta t$ is
$\vec p(t+\Delta t)-\vec p(t)$ and the change in \emph{distance} is $\norm{\vec p(t+\Delta t) - \vec p(t)}$.
Thus, if $\Delta t$ is small, the velocity at time $t$ can be approximated by
\[
		\Vel \vec p(t) \approx \frac{\vec p(t+\Delta t) - \vec p(t)}{\Delta t}	
\]
and the speed\footnote{
		Recall that speed is always positive; if a particle is moving with speed $2$
		and we then ran the particle back in time, it would still move at speed $2$,
		so speed is not $\text{distance}/\Delta t$, it is $\text{distance}/\abs{\Delta t}$.
		} by
\[
	\Speed \vec p(t) \approx \frac{\norm{\vec p(t+\Delta t) - \vec p(t)}}{\abs{\Delta t}}.	
\]
Taking limits, we arrive at exact rates of change, which leads us to the following definitions.
\begin{definition}[Speed]
	Let $\vec p:[a,b]\to\mathcal S$ be a parameterization of $\mathcal S$.  The
	\emph{speed} of $\vec p$ at the time $t\in[a,b]$ is 
	\[
		\Speed \vec p(t) = \lim_{\Delta t\to 0} \frac{\norm{\vec p(t+\Delta t) - \vec p(t)}}{\abs{\Delta t}}.	
	\]
\end{definition}
\begin{definition}[Velocity]
	Let $\vec p:[a,b]\to\mathcal S$ be a parameterization of $\mathcal S$.  The
	\emph{velocity} of $\vec p$ at the time $t\in[a,b]$ is 
	\[
		\Vel \vec p(t) = \lim_{\Delta t\to 0} \frac{\vec p(t+\Delta t) - \vec p(t)}{\Delta t}.	
	\]
\end{definition}

Both the definition of speed and the definition of velocity look a lot like the definition
of the derivative.  In fact, if $\vec p$ were a scalar valued function, the velocity of $\vec p$
would be exactly the derivative of $\vec p$.  For this reason, we will define a notation
similar to that of the derivative you're familiar with.  From now on, the following notations mean
the same thing:
\[
	\Vel \vec p(t) = \vec p\,'(t) = \frac{\d}{\d t}\vec p(t) = \frac{\d \vec p}{\d t}(t).
\]

Let's try to use our new definition.  Let $\vec r(t) = \mat{\cos t\\\sin t}$.  Now,
\begin{align*}
	\Vel \vec r(t) &=\lim_{\Delta t\to 0} \frac{\mat{\cos(t+\Delta t)\\\sin(t+\Delta t)} - \mat{\cos t\\\sin t}}{\Delta t}\\
	&=\lim_{\Delta t\to 0} \mat{
		\displaystyle\frac{\cos(t+\Delta t)-\cos t}{\Delta t} \\\displaystyle\frac{\sin(t+\Delta t)-\sin t}{\Delta t}
		}.
\end{align*}
At this point, we should pause.  We don't know how to take limits of vectors.  Fortunately the rule
is simple enough---to take a limit of a vector, take the limit of each of its components\footnote{
	As intuitive as it sounds, this rule actually has a proof which relies on the definition
	of limit and the continuity of $\norm{\:\cdot\:}$.
}.  Thus we see
\begin{align*}
	\Vel \vec r(t) 
	&=\lim_{\Delta t\to 0} \mat{
		\displaystyle\frac{\cos(t+\Delta t)-\cos t}{\Delta t} \\\frac{\displaystyle\sin(t+\Delta t)-\sin t}{\Delta t}}\\
	&=\mat{
		\displaystyle\lim_{\Delta t\to 0} \frac{\cos(t+\Delta t)-\cos t}{\Delta t} \\\displaystyle\lim_{\Delta t\to 0} \frac{\sin(t+\Delta t)-\sin t}{\Delta t}
		}\\
	&=\mat{\cos'(t)\\\sin'(t)} = \mat{-\sin t\\\cos t}.
\end{align*}
Our use of the notation $\vec r\,'(t)$ for $\Vel \vec r(t)$ seems further justified.

Speed also appears to be a derivative.  From physics, we know that speed is the magnitude of 
velocity.  We can prove it mathematically.
\begin{theorem}
	For a parameterization $\vec p:\R\to\R^n$ where $\Vel \vec p(t)$
	exists, we have
	\[
		\Speed \vec p(t) = \norm{\Vel \vec p(t)} = \norm{\vec p\,'(t)}.
	\]
\end{theorem}
\begin{proof}
	The proof relies on the continuity of $\norm{\:\cdot\:}$.  Since $\norm{\:\cdot\:}$
	is continuous, we may freely move limits in and out\footnote{
		This limit rule actually says if $f$ is continuous 
		\emph{and} $\lim_{x\to a} g(x)$ exists, then $\lim_{x\to a}f(g(x))
		=f\big(\lim_{x\to a}g(x)\big)$.}.  Thus
	\begin{align*}
		\Speed \vec p(t) &= \lim_{\Delta t\to 0} \frac{\norm{\vec p(t+\Delta t) - \vec p(t)}}{\abs{\Delta t}}\\
		&= \lim_{\Delta t\to 0} \norm*{\frac{\vec p(t+\Delta t) - \vec p(t)}{\Delta t}}\\
		&=\norm*{\lim_{\Delta t\to 0} \frac{\vec p(t+\Delta t) - \vec p(t)}{\Delta t}}
		=\norm{\Vel \vec p(t)}.
	\end{align*}
\end{proof}

\subsection{Arc-length}
Let $\mathcal S\subseteq \R^n$ be a curve parameterized by $\vec p:[a,b]\to\R^n$.  
The \emph{arc-length}\index{arc length} of $\mathcal S$
should be the length of $\mathcal S$ if you somehow untwisted $\mathcal S$ into a straight
line without stretching anything.  One of the big ideas of calculus is that 
we can handle curvy things by chopping them up into little pieces, computing
for each piece, and then adding them back together.  We use the same principle 
to define arc-length.

In essence, we will divide our curve $\mathcal S$ into many tiny line segments,
add up the lengths of those line segments and take a limit as our line segments
get tinier.  A parameterization provides us with a way to do this.  Since parameterizations
are continuous, if we chop the domain of the parameterization into tiny pieces,
we will have chopped the range into tiny pieces.


% Some nice looking curves:
% x=sin(t^.7)*(t/4-t^2/30)*10+2
% y=((10-t)^.5+t^2/10)*cos(t/10)
% [0,10]
%
% or
% x=-(t+1)*t*(t-3)*(1+t^2)/4
% y=-t*(t-1)*(t-2)*4*(1+t^3)/4
% [0,2]

\begin{center}
	\begin{tikzpicture}
		\begin{axis}[
		    name=plot1,
		    %anchor=origin,
		    %disabledatascaling,
		    xmin=-1,xmax=8,
		    ymin=-1,ymax=2.5,
		    x=1cm,y=1cm,
		    xtick=\empty,ytick=\empty,
		    grid=both,
		    grid style={line width=.1pt, draw=gray!10},
		    %major grid style={line width=.2pt,draw=gray!50},
		    axis lines=middle,
		    minor tick num=0,
		    enlargelimits={abs=0.5},
		    axis line style={latex-latex},
		    ticklabel style={font=\tiny,fill=white},
		    xlabel style={at={(ticklabel* cs:1)},anchor=north west},
		    ylabel style={at={(ticklabel* cs:1)},anchor=south west}
		]

		\addplot[domain=0:2, samples=80, thick, mypink] ({-(x+1)*x*(x-3)*(1+x^2)/4},{-x*(x-1)*(x-2)*4*(1+x^3)/4});
		\addplot[domain=0:2, samples=15, only marks, mark=|, gray, thick] ({-(x+1)*x*(x-3)*(1+x^2)/4},{-x*(x-1)*(x-2)*4*(1+x^3)/4});
		\addplot[domain=0:2, samples=15, gray,thick] ({-(x+1)*x*(x-3)*(1+x^2)/4},{-x*(x-1)*(x-2)*4*(1+x^3)/4});
		\draw[] (0,2.5) node[right,gray] {Approximation by Line Segments};
		\draw[] (0,.5) node[right] {Range};

		\end{axis}
		\begin{axis}[
		    at=(plot1.below south),anchor=above north,yshift=-.5cm,
		    %anchor=origin,
		    %disabledatascaling,
		    xmin=-1,xmax=8,
		    ymin=-.2,ymax=.2,
		    hide y axis,
			xtick=\empty,
		    x=1cm,y=1cm,
		    grid=both,
		    grid style={line width=.1pt, draw=gray!10},
		    %major grid style={line width=.2pt,draw=gray!50},
		    axis lines=middle,
		    minor tick num=0,
		    enlargelimits={abs=0.5},
		    axis line style={latex-latex},
		    ticklabel style={font=\tiny,fill=white},
		    xlabel style={at={(ticklabel* cs:1)},anchor=north west},
		    ylabel style={at={(ticklabel* cs:1)},anchor=south west}
		]

			\addplot[domain=0:6, samples=2, thick, mypink] ({x},{0});
		\addplot[domain=0:6, samples=15, only marks, mark=|, gray, thick] ({x},{0});
		\draw[] (0,.5) node[right] {Domain};

		\end{axis}
	\end{tikzpicture}
\end{center}

To find the arc-length of a curve, we will approximate each tiny piece with a straight
line segment connecting the endpoints.  We then add up the lengths of all tiny segments
and take a limit as our segment's length goes to zero.

\begin{definition}[Arc-length]
	Let $\mathcal S\subseteq \R^n$ be a curve parameterized by $\vec p:[a,b]\to\mathcal S$. 
	The \emph{arc-length}
	of $\mathcal S$ is 
	\[
		\Arclen\mathcal S = \lim_{\Delta t\to 0^+} \sum_{i=1}^{\frac{b-a}{\Delta t}} 
		\,\norm{\vec p(a+(i-1)\Delta t)-\vec p(a+i\Delta t)}.
	\]
\end{definition}

There's something unsatisfying about this definition, though.  We used a parameterization
of $\mathcal S$ to compute the arc-length of $\mathcal S$.  But $\mathcal S\subseteq \R^n$ 
is a curve regardless of whether or not it has a parameterization, and if you use a different
parameterization, you should get the same arc length for $\mathcal S$.  If you're worried about
this, good!  You're thinking carefully!  We won't show it here, but in fact no matter what parameterization
you use for a curve, this definition will always produce the same arc length.

There's another reason we might be unhappy with this definition.  Limits of sums are
hard to compute!  However, the sum involved in arc-length looks very close to a Riemann sum.
If we can rewrite it exactly as a Riemann sum, we can replace it with an integral.  With some
superficial manipulation we see

\begin{align*}
	\Arclen\mathcal S 
	&= \lim_{\Delta t\to 0^+} \sum_{i=1}^{\frac{b-a}{\Delta t}} 
	\,\norm{\vec p(a+(i-1)\Delta t)-\vec p(a+i\Delta t)}\\
	&= \lim_{\Delta t\to 0^+} \sum_{i=1}^{\frac{b-a}{\Delta t}} 
	\,\frac{\norm{\vec p(a+(i-1)\Delta t)-\vec p(a+i\Delta t)}}{\Delta t}\Delta t\\
	&=\int_{a}^b \Speed \vec p(t)\,\d t.
\end{align*}

For the last equality, we noticed that 
$\lim_{\Delta t\to 0^+} \frac{\norm{\vec p(t)-\vec p(t+\Delta t)}}{\Delta t}=\Speed \vec p(t)$,
which involved switching a limit and an infinite sum.  In order to do this rigorously, we need
a mathematical proof that it is logically valid.  Such a proof is the subject of a course in
\emph{real analysis}, and won't be covered here, but it's always good to keep track of what
you've actually proved and what you've been told is true\footnote{ In order to prove that swapping
the limit and sum is valid, we actually need extra assumptions on $\vec p$.  If we make $\vec p$
\emph{differentiable} rather than merely \emph{continuous}, we can prove that the swap is valid.}.

Speed is easy to calculate, and we have a better handle on calculating integrals than we do 
limits of sums, so now we have a chance of calculating arc-length.

\begin{example}
	\label{EXHARDARCLEN}
We shall find the length of the parabola with equation $y = x^2$
on the interval $-1\le x \le 1$.   A parametric representation
	of the parabola is $\vec p(t) = (t,t^2)$
where $-1 \le t \le 1$.  
Now,
\[
	\frac{\d\vec p}{\d t} = \mat{1\\2t},
\]
so $\Speed \vec p(t)=\norm{\vec p\,'(t)} = \sqrt{ 1 + 4t^2}$.
Hence,
\[
  L = \int_{-1}^1 \sqrt{1 + 4t^2} dt = \sqrt 5 + 
\frac 14\ln\frac{\sqrt 5 + 2}{\sqrt 5 - 2}.
\]
\end{example}

As you can see from Example \ref{EXHARDARCLEN}, the integrals involved in computing
arc length can be difficult.  In fact, most of them don't have an elementary form, which
means in the real world we often use a computer to approximate arc length directly from the Riemann sum
rather than calculate it exactly.

\begin{exercises}
\end{exercises}

\section{Arc-length Parameterization}

Recall a parameterization is a relation between two objects.  If a curve $\mathcal S\subseteq\R^n$
is parameterized by $\R$, it means there is a continuous, one-to-one function $\vec p:\R\to\mathcal S$.
This function can be thought of as a map from $\R$ to $\mathcal S$.  Any interval $[a,b]\subseteq\R$
corresponds to a segment $\vec p([a,b])\subseteq \mathcal S$.

\begin{center}
	\usetikzlibrary{patterns,decorations.pathreplacing}
	\begin{tikzpicture}
		\begin{axis}[
		    name=plot1,
		    %anchor=origin,
		    %disabledatascaling,
		    xmin=-1,xmax=8,
		    ymin=-1,ymax=2.5,
		    x=1cm,y=1cm,
		    xtick=\empty,ytick=\empty,
		    grid=both,
		    grid style={line width=.1pt, draw=gray!10},
		    %major grid style={line width=.2pt,draw=gray!50},
		    axis lines=middle,
		    minor tick num=0,
		    enlargelimits={abs=0.5},
		    axis line style={latex-latex},
		    ticklabel style={font=\tiny,fill=white},
		    xlabel style={at={(ticklabel* cs:1)},anchor=north west},
		    ylabel style={at={(ticklabel* cs:1)},anchor=south west}
		]
		
		\pgfmathsetmacro\mystart{1.2}
		\pgfmathsetmacro\myend{1.7}
		\coordinate (A) at ($({-(\mystart+1)*\mystart*(\mystart-3)*(1+\mystart^2)/4},{-\mystart*(\mystart-1)*(\mystart-2)*4*(1+\mystart^3)/4})$);
		\coordinate (B) at ($({-(\myend+1)*\myend*(\myend-3)*(1+\myend^2)/4},{-\myend*(\myend-1)*(\myend-2)*4*(1+\myend^3)/4})$);

		\addplot[domain=0:2, samples=80, thick, mypink] ({-(x+1)*x*(x-3)*(1+x^2)/4},{-x*(x-1)*(x-2)*4*(1+x^3)/4});
		\addplot[domain=\mystart:\myend, samples=20, ultra thick, blue] ({-(x+1)*x*(x-3)*(1+x^2)/4},{-x*(x-1)*(x-2)*4*(1+x^3)/4});
		\draw[] (0,.5) node[right] {Range};
		
		\draw[fill] (A) circle (1.5pt) node[above left] {$\vec p(a)$};
		\draw[fill] (B) circle (1.5pt) node[above] {$\vec p(b)$};

		\end{axis}
		\begin{axis}[
		    at=(plot1.below south),anchor=above north,yshift=-.5cm,
		    %anchor=origin,
		    %disabledatascaling,
		    xmin=-1,xmax=8,
		    ymin=-.2,ymax=.2,
		    hide y axis,
			xtick=\empty,
		    x=1cm,y=1cm,
		    grid=both,
		    grid style={line width=.1pt, draw=gray!10},
		    %major grid style={line width=.2pt,draw=gray!50},
		    axis lines=middle,
		    minor tick num=0,
		    enlargelimits={abs=0.5},
		    axis line style={latex-latex},
		    ticklabel style={font=\tiny,fill=white},
		    xlabel style={at={(ticklabel* cs:1)},anchor=north west},
		    ylabel style={at={(ticklabel* cs:1)},anchor=south west}
		]

		\addplot[domain=0:6, samples=2, thick, mypink] ({x},{0});
		\draw[] (0,.5) node[right] {Domain};
		\coordinate (A) at (3,0);
		\coordinate (B) at (5,0);
		\draw[ultra thick, blue] (A) -- (B);
		\draw[fill] (A) circle (1.5pt)  node[above] {$a$};
		\draw[fill] (B) circle (1.5pt)  node[above] {$b$};

		\end{axis}
	\end{tikzpicture}
\end{center}

Alternatively, we may think of $\vec p:\R\to\mathcal S$ as a function that picks up the real line,
stretches, twists, and warps it, and sticks it into $\R^n$ in the shape of $\mathcal S$.
In this sense, not all parameterizations are created equally.  Some significantly stretch and warp
and others barely do at all.  

\begin{center}
	\usetikzlibrary{patterns,decorations.pathreplacing}
	\begin{tikzpicture}
	\pgfmathsetmacro{\samplestep}{4}
		\begin{axis}[
		    name=plot1,
		    %anchor=origin,
		    %disabledatascaling,
		    xmin=-1,xmax=8,
		    ymin=-1,ymax=2.5,
		    x=1cm,y=1cm,
		    xtick=\empty,ytick=\empty,
		    grid=both,
		    grid style={line width=.1pt, draw=gray!10},
		    %major grid style={line width=.2pt,draw=gray!50},
		    axis lines=middle,
		    minor tick num=0,
		    enlargelimits={abs=0.5},
		    axis line style={latex-latex},
		    ticklabel style={font=\tiny,fill=white},
		    xlabel style={at={(ticklabel* cs:1)},anchor=north west},
		    ylabel style={at={(ticklabel* cs:1)},anchor=south west}
		]

		\draw[] (0,.5) node[right] {Range};
		

			\addplot[domain=0:2, samples=80, ultra thick, myorange,
			    x filter/.code={%
				\pgfmathsetmacro{\orig}{\pgfmathresult}
				\pgfmathsetmacro{\modcoord}{mod(int(mod(\coordindex, 2*\samplestep)),2*\samplestep)}
				\pgfmathparse{\modcoord <= \samplestep ? \orig : nan}
			    }, unbounded coords=jump] ({-(x+1)*x*(x-3)*(1+x^2)/4},{-x*(x-1)*(x-2)*4*(1+x^3)/4});
			\addplot[domain=0:2, samples=80, ultra thick, blue,
			    x filter/.code={%
				\pgfmathsetmacro{\orig}{\pgfmathresult}
				\pgfmathsetmacro{\modcoord}{mod(int(mod(\coordindex, 2*\samplestep)),2*\samplestep)}
				\pgfmathparse{\modcoord == 0 || \modcoord >= \samplestep ? \orig : nan}
			    }, unbounded coords=jump] ({-(x+1)*x*(x-3)*(1+x^2)/4},{-x*(x-1)*(x-2)*4*(1+x^3)/4});
		\end{axis}
		\begin{axis}[
		    at=(plot1.below south),anchor=above north,yshift=-.5cm,
		    %anchor=origin,
		    %disabledatascaling,
		    xmin=-1,xmax=8,
		    ymin=-.2,ymax=.2,
		    hide y axis,
			xtick=\empty,
		    x=1cm,y=1cm,
		    grid=both,
		    grid style={line width=.1pt, draw=gray!10},
		    %major grid style={line width=.2pt,draw=gray!50},
		    axis lines=middle,
		    minor tick num=0,
		    enlargelimits={abs=0.5},
		    axis line style={latex-latex},
		    ticklabel style={font=\tiny,fill=white},
		    xlabel style={at={(ticklabel* cs:1)},anchor=north west},
		    ylabel style={at={(ticklabel* cs:1)},anchor=south west}
		]
			\addplot[domain=0:6, samples=81, ultra thick, myorange,
			    x filter/.code={%
				\pgfmathsetmacro{\orig}{\pgfmathresult}
				\pgfmathsetmacro{\modcoord}{mod(int(mod(\coordindex, 2*\samplestep)),2*\samplestep)}
				\pgfmathparse{\modcoord <= \samplestep ? \orig : nan}
			    }, unbounded coords=jump] ({x},{0});
			\addplot[domain=0:6, samples=81, ultra thick, blue,
			    x filter/.code={%
				\pgfmathsetmacro{\orig}{\pgfmathresult}
				\pgfmathsetmacro{\modcoord}{mod(int(mod(\coordindex, 2*\samplestep)),2*\samplestep)}
				\pgfmathparse{\modcoord == 0 || \modcoord >= \samplestep ? \orig : nan}
			    }, unbounded coords=jump] ({x},{0});

		\draw[] (0,.5) node[right] {Domain};

		\end{axis}
	\end{tikzpicture}
\end{center}


The least stretchy type of parameterization is called an \emph{arc-length
parameterization}\index{arc-length parameterization}.

Before we define arc-length parameterization, let's introduce some notation.  If $\mathcal S\subseteq \R^n$
is a curve parameterized by $\vec p:\R\to\mathcal S$, then
\[
	\Arclenfrom{\vec p}{a}{b}  = \text{ arc length of $\mathcal S$ between $\vec p(a)$ and $\vec p(b)$}.
\]
We might read $\Arclenfrom{\vec p}{a}{b}$ as the ``arc length of the curve traced by $\vec p(t)$ from
$t=a$ to $t=b$.''  Using notion for the \emph{image} of a set, we can also write
\[
	\Arclenfrom{\vec p}{a}{b} = \Arclen \vec p([a,b]).
\]

\begin{definition}
	Let $\mathcal S\subseteq\R^n$ be a curve and $\vec p:\R\to\mathcal S$ be a parameterization.
	The parameterization $\vec p$ is called an \emph{arc-length parameterization} if for $b\geq a$,
	\[
		\Arclenfrom{\vec p}{a}{b}=b-a.
	\]
\end{definition}

In plain language, if $\vec p$ is an arc-length parameterization, then the distance traveled in parameter
space is the same as the distance traveled along the curve.

\begin{center}
	\usetikzlibrary{patterns,decorations.pathreplacing}
	\begin{tikzpicture}
	\pgfmathsetmacro{\samplestep}{4}
		\begin{axis}[
		    name=plot1,
		    %anchor=origin,
		    %disabledatascaling,
		    xmin=-1,xmax=8,
		    ymin=-1,ymax=2.5,
		    x=1cm,y=1cm,
		    xtick=\empty,ytick=\empty,
		    grid=both,
		    grid style={line width=.1pt, draw=gray!10},
		    %major grid style={line width=.2pt,draw=gray!50},
		    axis lines=middle,
		    minor tick num=0,
		    enlargelimits={abs=0.5},
		    axis line style={latex-latex},
		    ticklabel style={font=\tiny,fill=white},
		    xlabel style={at={(ticklabel* cs:1)},anchor=north west},
		    ylabel style={at={(ticklabel* cs:1)},anchor=south west}
		]

		\draw[] (0,.5) node[right] {Range};
		

			\addplot[domain=0:2, samples=80, ultra thick, myorange,
				dash pattern=on 8pt off 8pt
			    ] ({-(x+1)*x*(x-3)*(1+x^2)/4},{-x*(x-1)*(x-2)*4*(1+x^3)/4});
			\addplot[domain=0:2, samples=80, ultra thick, blue,
				dash pattern=on 8pt off 8pt, dash phase=8pt
			    ] ({-(x+1)*x*(x-3)*(1+x^2)/4},{-x*(x-1)*(x-2)*4*(1+x^3)/4});
		\end{axis}
		\begin{axis}[
		    at=(plot1.below south),anchor=above north,yshift=-.5cm,
		    %anchor=origin,
		    %disabledatascaling,
		    xmin=-1,xmax=8,
		    ymin=-.2,ymax=.2,
		    hide y axis,
			xtick=\empty,
		    x=1cm,y=1cm,
		    grid=both,
		    grid style={line width=.1pt, draw=gray!10},
		    %major grid style={line width=.2pt,draw=gray!50},
		    axis lines=middle,
		    minor tick num=0,
		    enlargelimits={abs=0.5},
		    axis line style={latex-latex},
		    ticklabel style={font=\tiny,fill=white},
		    xlabel style={at={(ticklabel* cs:1)},anchor=north west},
		    ylabel style={at={(ticklabel* cs:1)},anchor=south west}
		]
			\addplot[domain=-1:8, samples=80, ultra thick, myorange,
				dash pattern=on 8pt off 8pt
			    ] ({x},{0});
			\addplot[domain=-1:8, samples=80, ultra thick, blue,
				dash pattern=on 8pt off 8pt, dash phase=8pt
			    ] ({x},{0});

		\draw[] (0,.5) node[right] {Domain};

		\end{axis}
	\end{tikzpicture}
\end{center}


\begin{example}
	Find an arc-length parameterization of the line $\ell$ which passes through
	the origin and has a direction vector $\vec d=\xhat+2\yhat+\zhat$.

	Let's name our parameterization $\vec p$ and see if we can build up
	a formula for $\vec p$.
	Since $\ell$ passes through the origin, let's have $\vec p(0)=\vec 0$.
	Now, $\vec p(1)$ has to be a point on $\ell$ that is distance $1$ from $\vec 0$.
	There are two such points, so we'll arbitrarily declare
	$
		\vec p(1) = \frac{1}{\sqrt{6}}(1,2,1).
	$
	The point $\vec p(2)$ must be distance $1$ from $\vec p(1)$ and distance $2$ from $\vec p(0)$.
	Since parameterizations aren't allowed to double-back, we only have one choice.  Namely,
	$
		\vec p(2) = \frac{2}{\sqrt{6}}(1,2,1).
	$
	Noticing the pattern, let's guess
	\[
		\vec p(t) = \frac{t}{\sqrt{6}}\mat{1\\2\\1}.
	\]
	Now we'll verify that $\vec p$ is an arc-length parameterization.  Since $\ell$ is
	a straight line this is easy:
	\[
		\Arclenfrom{\vec p}{a}{b}=\norm{\vec p(b)-\vec p(a)} = \abs*{\tfrac{b-a}{\sqrt{6}}}\norm*{\mat{1\\2\\1}}
		=\abs{b-a},
	\]
	and so $\vec p$ is an arc-length parameterization.
\end{example}

Arc-length parameterizations can also be characterized by their speed.  For an arc-length
parameterization, the distance traveled in parameter space must be equal to the distance
traveled along the curve.  Therefore, an arc-length parameterization must also move at
speed 1.

\begin{theorem}
	Let $\mathcal S\subseteq\R^n$ be a curve and $\vec p:\R\to\mathcal S$
	a parameterization.  The parameterization $\vec p$ is an arc-length
	parameterization if and only if $\Speed\vec p=1$.
\end{theorem}
\begin{proof}
	Suppose $\vec p$ is a parameterization of $\mathcal S$ and $\Speed \vec p=1$.  Then
	\[
		\Arclenfrom{\vec p}{a}{b} = \int_a^b\Speed\vec p(t)\,\d t = \int_a^b1\d t = b-a.
	\]

	Now, suppose that $\vec p$ satisfies $\Arclenfrom{\vec p}{a}{b}=b-a$.  We then know if
	$\Delta t>0$ is small
	\[
		\frac{\norm{\vec p(t+\Delta t)-\vec p(t)}}{\Delta t} \approx \frac{\Arclenfrom{\vec p}{t}{t+\Delta t}}{\Delta t}
		= \frac{(t+\Delta t) - t}{\Delta t} = 1.
	\]
	After taking a limit as $\Delta t\to 0$, ``$\approx$'' will turn into ``$=$''.

	To make this argument completely rigorous, we will have to do a little bit of
	mathematical gymnastics.  Since the shortest distance between two points is a straight
	line, what we really know is
	\[
		\norm{\vec p(t+\Delta t)-\vec p(t)} \leq \Arclenfrom{\vec p}{t}{t+\Delta t}
		 = \Delta t
	\]
	for all $\Delta t$.  Now, taking a limit as $\Delta t\to 0$, we deduce $\Speed \vec p(t)\leq 1$.  But,
	\[
		b-a = \Arclenfrom{\vec p}{a}{b} = \int_a^b \Speed \vec p(t)\,\d t \leq
		\int_a^b1\d t  = b-a,
	\]
	and so if $\Speed \vec p$ is a continuous function, $\Speed\vec p(t)=1$ for all $t$.
	Even if speed is not continuous, we can argue that $\Speed \vec p$ is \emph{essentially} 1.\footnote{
	Here, the word \emph{essentially} is a technical term coming from real analysis.
		}
\end{proof}

Now we have a new way of thinking about arc-length parameterizations and a new way of creating them.
Instead of working with arc length directly, we can attempt to manipulate \emph{speed}.

\begin{example}
	\label{EXARCLENPARAM}
	Find an arc-length parameterization of $\mathcal C$, the circle of radius $2$ centered at the origin.

	We already know a parameterization of $\mathcal C$.  Namely,
	\[
		\vec r(t) =\mat{2\cos t\\2\sin t}.
	\]
	However, $\Speed \vec r(t) =\norm{\vec r\,'(t)} = \sqrt{(-2\sin t)^2+(2\cos t)^2} = 2$.  If we could
	somehow \emph{slow down} time, we could slow down the speed to be $1$, giving an arc-length parameterization.

	Let $w(t)=t/2$.  The function $w$ stretches time by a factor of $2$.  Define
	\[
		\vec p(t) = \vec r\circ w(t) = \mat{2\cos t/2\\2\sin t/2}.
	\]
	Now, 
	\begin{align*}
		\Speed \vec p(t) = \norm{\vec p\,'(t)}&=\norm{(\vec r\circ w)'(t)} = \norm{(\vec r\,'\circ w(t))w'(t)}\\
		&=\abs{w'(t)}\norm{\vec r\,'\circ w(t)} = \tfrac{1}{2}2=1.
	\end{align*}

	In this computation we made judicious use of the chain rule, however we could have computed
	directly from our formula for $\vec p$.  Now,
	since $\Speed\vec p(t)=1$, the function $\vec p$ is an arc-length parameterization.
\end{example}

\begin{exercise}
	\label{EXCANTMUL}
	Let $\mathcal C$ and $\vec r$ be as in Example \ref{EXARCLENPARAM}.  The function $\vec q(t)=\tfrac{1}{2}\vec r(t)$
	has speed 1 and so is an arc-length parameterization of something.  Explain why $\vec q$ is \emph{not}
	an arc-length parameterization of $\mathcal C$.
\end{exercise}

In Example \ref{EXARCLENPARAM}, we adjusted the speed of a parameterization by warping time.  Suppose
$\mathcal S\subseteq \R^n$ is a curve parameterized by $\vec p:\R\to \mathcal S$ and let $w:\R\to\R$ be 
a parameterization of $\R$.  In other words, $w$ stretches or squishes (or flips) $\R$ by varying amounts.
Now consider
\[
	\vec r=\vec p\circ w.
\]
The function $\vec r$ has domain $\R$ and range $\mathcal S$.  Further, since $\vec p$ and $w$ are both
one-to-one and continuous, we know $\vec r$ is one-to-one and continuous.  Thus, $\vec r$ is a parameterization
of $\mathcal S$.  And, by using the chain rule,
\[
	\vec r\,'(t) = (\vec p\circ w)'(t) = w'(t)\big[\vec p\,'\circ \vec w(t)\big],
\]
so
\[
	\Speed\vec r\text{ at time $t$} = \abs{w'(t)}\big[\Speed \vec p\text{ at time $w(t)$}\big].
\]
It is important to note that we must compose $\vec p$ and $w$ in order to get a parameterization of $\mathcal S$.
See Exercise \ref{EXCANTMUL} for an example of why you cannot multiply $\vec p$ and $w$.

With the idea of stretching time in the back of our minds, let's work through a hypothetical example.

Suppose $\mathcal S\subseteq \R^n$ is a curve parameterized by $\vec p:\R\to\mathcal S$ and we've computed
the following table of values.
\begin{center}
	\begin{tabular}{c|c}
		$t$ & $\Arclenfrom{\vec p}{0}{t}$\\
		\hline
		$0$ & $0$\\
		$1$ & $2$\\
		$2$ & $3.5$\\
		$3$ & $6$\\
		$4$ & $7$
	\end{tabular}
\end{center}

XXX Figure

We'd like to find a time-stretching function $w:\R\to\R$ so that $\vec r=\vec p\circ w$ is an arc-length
parameterization of $\mathcal S$.  In other words, we need
\[
	\Arclenfrom{\vec r}{0}{t} = \Arclenfrom{(\vec p\circ w)}{0}{t}=t.
\]
In particular, $\Arclenfrom{\vec r}{0}{0}=0$ (this we get for free) and $\Arclenfrom{\vec r}{0}{2}=2$.
From the table, we know that the arc length from $\vec p(0)$ to $\vec p(1)$ is $2$, and so we need
$w(2) =1$.  This way $\vec r(2) = \vec p(w (2)) = \vec p(1)$.  Continuing in this way, we get the following
table of values for $w$.
\begin{center}
	\begin{tabular}{c|c}
		$t$ & $w(t)$\\
		\hline
		$0$ & $0$\\
		$2$ & $1$\\
		$3.5$ & $2$\\
		$6$ & $3$\\
		$7$ & $4$
	\end{tabular}
\end{center}

The function $w$ is just the inverse of the function $\Arclenfrom{\vec p}{0}{t}$!  This also makes sense
from a purely algebraic perspective.  Consider
\[
	x=\Arclenfrom{\vec r}{0}{x} = \Arclenfrom{(\vec p\circ w)}{0}{x}=\Arclenfrom{(\vec p)}{0}{w(x)}.
\]
Replacing $x$ with $w^{-1}(t)$, we see
\[
	w^{-1}(t) = \Arclenfrom{\vec p}{0}{w\circ w^{-1}(t)} = \Arclenfrom{\vec p}{0}{t}.
\]
This means the inverse of $w$ is $\Arclenfrom{\vec p}{0}{t}$ and so the inverse of $\Arclenfrom{\vec p}{0}{t}$
must be $w$.\footnote{ Recall that for an invertible function $f$, we have $(f^{-1})^{-1}=f$.}

We now have a concrete way to find an arc-length parameterization.
\begin{example}
	Let $\mathcal R$ be the ray parameterized by 
	$\vec p:[0,\infty)\to\R^2$ where $\vec p(t) = (t^2,2t^2)$.  Find an arc-length
	parameterization of $\mathcal R$.

	We'll start by finding the arc-length function for $\vec p$.
	\begin{align*}
		a(t)=\Arclenfrom{\vec p}{0}{t} &= \int_0^t \norm{\vec p\,'(x)}\d x\\
		&=\int_0^t \sqrt{(2x)^2+(4x)^2}\d x = t^2\sqrt{5}.
	\end{align*}
	Now define $w(t)=a^{-1}(t)$.  Since $t\geq 0$, the inverse of $a$ is well defined
	and is given by
	\[
		w(t)=a^{-1}(t) = \sqrt{t/\sqrt{5}}.
	\]
	Now, define $\vec r(t) = \vec p\circ w(t)=(t/\sqrt{5},2t/\sqrt{5})$.  The parameterization
	$\vec r$ is an arc-length parameterization!

	Of course we didn't need to go through all this work in this case.  If all we wanted was
	to find an arc-length parameterization of $\mathcal R$ we could create one directly using
	geometry, since we know how to parameterize lines and rays.
\end{example}

\subsection{Explicit Arc-length Parameterizations}
The idea of arc-length parameterization is very important.  However, for most curves,
we're hopeless in finding a formula for the arc-length parameterization.  We can for a hand
full of curves, like a circle, a line, a helix, but even something as simple as an ellipse
cannot be arc-length parameterized with elementary functions.

Why is it so hard?  Well, integrals are hard in general.  Most formulas cannot be integrated in closed
form.  Integrals involving square roots are even harder to evaluate.  And, even if you manage
to integrate to find the arc-length function, you still have to invert that function\footnote{ If you don't
believe me, go ahead and 
try to find the inverse of the function $f(x)=xe^x$. }.  So, if finding formulas for arc-length parameterizations
is so hard, why do we bother with them at all?  The answer is that the \emph{idea} of an arc-length parameterization
is incredibly useful. Its mere existence will aid our thinking.  And, in many of the problems we will be solving,
the arc-length parameterization will somehow get canceled out and we won't ever need to find a formula for it.

\begin{exercises}
\end{exercises}

\section{Acceleration and Curvature}
In the Newtonian mechanics of one-dimensional motion, acceleration\index{acceleration} is the second
derivative of position with respect to time.  In $\R^n$ we define it in the same way.

\begin{definition}[Acceleration]
	Let $\vec p:\R\to\mathcal S$ be a parameterization of $\mathcal S$.  The \emph{acceleration}
	of $\vec p$ is 
	\[
		\Accel \vec p(t) = (\Vel \vec p)'(t)=\vec p\,''(t).
	\]
\end{definition}

Just like velocity, acceleration is a \emph{vector}.  Let's consider two examples.  Define
\[
	\vec l(t) = \mat{t^2\\t^2}\qquad\text{and}\qquad\vec c(t) = \mat{\cos t\\\sin t}.
\]
Here $\vec l$ parameterizes a ray and $\vec c$ a circle.  Further, $\vec l$ traces along the ray
faster and faster, whereas $\vec c$ has constant speed as it traces the circle.
We compute 
\[
	\Accel \vec l(t) = \mat{2\\2}\qquad\text{and}\qquad \Accel\vec c(t) = \mat{-\cos t\\ -\sin t},
\]
and see that the acceleration of $\vec l$ is constant whereas the acceleration of $\vec c$ is not.  Further,
$\norm{\Accel \vec l(t)}=2$ and $\norm{\Accel \vec c(t)}=1$, and so the magnitude of the acceleration of
both $\vec l$ and $\vec c$ is constant.

In the past, you might have distinguished linear acceleration (running faster and faster along a straight line) from
centripetal acceleration (the acceleration you experience by moving at a constant speed
around a circle).  With vectors, these two types of acceleration are unified into a single vector.  

In the previous example, $\vec l$ had purely linear acceleration and
the acceleration vector pointed tangent to the path $\vec l$ traced.
Analyzing $\vec c$, we see $\vec c$ had purely centripetal acceleration and the acceleration was orthogonal
to the curve it traced.
What happens if we mix the two types of acceleration?

Consider
\[
	\vec r(t)=\mat{\cos t^2\\\sin t^2}.
\]
Computing, 
\[
	\Accel \vec r(t) = \mat{-2(\sin t^2+2t^2\cos t^2)\\ 2(\cos t^2-2t^2\sin t^2)}.
\]
There is no clear relationship between the curve $\vec r$ traces and the acceleration vector
for $\vec r$.  To see this relationship, we need to decompose $\Accel \vec r$ into its tangential
and normal components\index{normal acceleration}\index{tangential acceleration}.

\begin{definition}[Tangential and Normal Acceleration]
	Let $\vec p:\R\to\mathcal S$ be a parameterization of $\mathcal S$.  Then
	$\Accel \vec p$ can be written as
	\[
		\Accel \vec p(t) = \vec a_T(t)+\vec a_N(t)
	\]
	where $\vec a_T(t)$ is tangent to $\mathcal S$ at the point $\vec p(t)$ and $\vec a_N(t)$
	is orthogonal to $\mathcal S$ at the point $\vec p(t)$.  In this case, $\vec a_T(t)$ is
	called the \emph{tangential component of the acceleration} and $\vec a_N(t)$ is
	called the \emph{normal component of the acceleration} of $\vec p$.
\end{definition}

\begin{example}
	Let $\vec r(t)=(\cos t^2,\sin t^2)$.  Find the tangential and normal components of
	the acceleration of $\vec r$.

	Earlier we computed
	\[
		\Accel \vec r(t) = \mat{-2(\sin t^2+2t^2\cos t^2)\\ 2(\cos t^2-2t^2\sin t^2)}.
	\]
	We can use projections to split $\Accel \vec r$ into its tangential and normal components.

	Recall $\vec r\,'(t)$ is tangent to the curve $\vec r$ traces at the point $\vec r(t)$.  Thus,
	\[
		\vec a_T = \Proj_{\vec r\,'(t)} \Accel \vec r(t) = \mat{-2\sin t^2\\ 2\cos t^2},
	\]
	and 
	\[
		\vec a_N = \Accel\vec r(t) - \vec a_T(t) = \mat{-4t^2\cos t^2\\ -4t^2\sin t^2}.
	\]
\end{example}

Suppose now that $\vec r:\R\to\mathcal S$ is an arc-length parameterization of $\mathcal S$.  Since
the speed of $\vec r$ is constant, it is intuitive that the tangential component of 
the acceleration of $\vec r$ is zero.  Equipped with our knowledge of vectors, it won't be
so hard to prove our intuition.  But, it will be helpful to establish the product rule
for dot products.

\begin{exercise}
	Let $\vec a(t)=(a_x(t),a_y(t),a_z(t))$ and $\vec b(t)=(b_x(t),b_y(t),b_z(t))$ be parameterizations.
	Establish the \emph{product rule for dot products}\index{product rule for dot products}.  That is,
	show that
	\[
		\Big[\vec a(t)\cdot \vec b(t)\Big]' = \vec a\,'(t)\cdot \vec b(t)+\vec a(t)\cdot \vec b\,'(t).
	\]
\end{exercise}

\begin{theorem}
	\label{THMARCLENACCEL}
	If $\vec p:\R\to\mathcal S$ is an arc-length parameterization of $\mathcal S$ then
	$\Accel\vec p$ is always orthogonal to $\mathcal S$.  Equivalently,
	\[
		\vec p\,''(t)\cdot \vec p\,'(t)=0.
	\]
\end{theorem}
\begin{proof}
	Since $\vec p$ is an arc-length parameterization, we know 
	\[
		\sqrt{\vec p\,'(t)\cdot \vec p\,'(t)} = \norm{\vec p\,'(t)}=1.
	\]
	Squaring both sides we get the relationship
	\begin{equation}
		\label{EQARCLENACCEL}
		\vec p\,'(t)\cdot \vec p\,'(t)=1.
	\end{equation}
	Now we may take the derivative of both sides of Equation \eqref{EQARCLENACCEL} and apply the
	product rule for dot products to find
	\[
		0=\big[\vec p\,'\cdot \vec p\,'\big]'(t) = \vec p\,''(t)\cdot \vec p\,'(t)+
		\vec p\,'(t)\cdot \vec p\,''(t) = 2\vec p\,''(t)\cdot \vec p\,'(t),
	\]
	and so $\vec p\,''(t)$ and $\vec p\,'(t)$ are orthogonal.
\end{proof}

If you examine the proof of theorem \ref{THMARCLENACCEL} closely, you'll notice that we didn't
actually need the speed of our parameterization to be $1$.  The proof works just as well if
the speed is some other constant.

\subsection{Curvature}
For any curve $\mathcal S$ there are infinitely many choices of parameterizations, but in some
sense, there is only one arc-length parameterization.  An arc-length parameterization of
$\mathcal S$ is uniquely determined by a direction (forwards or backwards along $\mathcal S$)
and a starting position.  Thus, we might think of an arc-length parameterization as 
\emph{intrinsic} to a curve.

The \emph{curvature}\index{curvature} of a curve is a measure of how sharply a curve bends or
twists.  Curvature is another property of a curve---you don't need a parameterization to define
curvature---but it is a lot easier to define with reference to an arc-length parameterization.

\begin{definition}[Curvature]
	Let $\mathcal S\subseteq\R^n$ be a curve and let $\vec p:\R\to\mathcal S$ be an arc-length
	parameterization of $\mathcal S$.  The \emph{curvature} of $\mathcal S$ at the point
	$\vec p(t)$ is 
	\[
		\norm{\Accel \vec p(t)} = \norm{\vec p\,''(t)}.
	\]
\end{definition}

This definition of curvature can be made intuitive.  If $\vec p:\R\to\mathcal S$ is an arc-length parameterization,
all velocity vectors are unit length.  Therefore, all acceleration of $\vec p$ must come from
the velocity vectors changing direction (and not changing length).  If a curve has a sharp bend (high curvature),
the velocity vectors with rapidly change direction.  If a curve is generally flat (low curvature),
the velocity vectors hardly change direction at all.

XXX Figure

Curvature can be hard to calculate exactly, but it isn't so hard to eyeball.

\begin{example}
	Estimate the curvature at various points of the parabola $y=x^2$.

	XXX Finish and include numerics
\end{example}

\begin{exercises}
\end{exercises}

\section{Line Integrals}

\section{Multi-dimensional Parameterizations}

Parameterizations aren't just for curves---higher dimensional objects can also be parameterized.
For example, let $\mathcal P\subseteq \R^3$ be the plane with equation $x+y+z=0$.  We can write this plane
in vector form as 
\[
	\vec x = t\mat{1\\-1\\0} + s\mat{0\\1\\-1}.	
\]
Written in vector form, we see that $\mathcal P$ is the range of a function $\vec p:\R^2\to\mathcal P$
defined by 
\[
	\vec p(t,s) = t\mat{1\\-1\\0} + s\mat{0\\1\\-1}.
\]
We could further verify that $\vec p$ is one-to-one and continuous,  
making $\vec p$ a parameterization
of $\mathcal P$.

Just like parameterizations of curves, for any given multi-dimensional object, there are
infinitely many parameterizations.  For instance,
\[
	\vec q(t,s) = t^3\mat{1\\-1\\0} + (1-s)\mat{0\\1\\-1}\qquad\text{and}\qquad
	\vec h(t,s) = t\mat{3\\-3\\0} + s\mat{1\\1\\-2}
\]
are also both parameterizations of $\mathcal P$.

\subsection{Converting Between Parameterizations}
Suppose $\vec p:\R\to\mathcal C$ and $\vec q:\R\to\mathcal C$ are both parameterizations of
the same curve $\mathcal C$ and that $\vec p(0)=\vec q(0)$.  Based on our previous explorations,
we know that we can convert from $\vec p$ to $\vec q$ by stretching, warping, or flipping ``time''
in some way.  That is, we can find a function $w:\R\to\R$ so that $\vec p\circ w=\vec q$.

Such time-warping functions, which we should really call \emph{domain}-warping
functions,
are themselves parameterizations of $\R$.  We can also
convert between parameterizations of multi-dimensional objects by conjuring up the
appropriate domain-warping function.

\begin{theorem}
	If $\vec p:\R^n\to\mathcal S$ and $\vec q:\R^n\to\mathcal S$ are both
	parameterizations of an $n$-dimensional object $\mathcal S$, then there exists a parameterization
	$\vec w:\R^n\to\R^n$ so that 
	\[
		\vec p\circ \vec w = \vec q.
	\]
\end{theorem}
\begin{proof}
	First, recall that the composition of two continuous functions is continuous.  Further,
	if a function $f:\R^n\to\R^m$ is continuous and invertible, then $f^{-1}$ is also continuous.

	Now, since $\vec p:\R^n\to\mathcal S$ is one-to-one, $\vec p^{-1}\Big|_{\mathcal S}$ exists
	and so $\vec w:\R^n\to\R^n$ given by $\vec w=\vec p^{\,-1}\circ\vec q$ is well defined.  Further, $\vec w$
	is a composition of parameterizations and therefore a parameterization.  It now directly follows that
	$\vec p\circ \vec w= \vec p\circ(\vec p^{-1}\circ \vec q) = \vec q$.
\end{proof}

\begin{example}
	Let $\mathcal P$ be a plane parameterized by $\vec p:\R^2\to\mathcal P$ and $\vec q:\R^2\to\mathcal P$.
	Given that
	\[
		\vec p(t,s) =  t\mat{1\\-1\\0} + s\mat{0\\1\\-1}\qquad\text{and}\qquad
		\vec q(t,s) =  t^3\mat{1\\-1\\0} + s\mat{1\\1\\-2},
	\]
	find a function $\vec w:\R^2\to\R^2$ such that $\vec p\circ \vec w=\vec q$.

	 We will find $\vec w$ by analyzing how it needs to adjust each of its parameters.
	Suppose $\vec w(t,s)=(a,b)$.  Then, if $\vec p\circ\vec w = \vec q$, we must have
	\[
		\vec p\circ \vec w(t,s) = \vec p(a,b) = a\mat{1\\-1\\0}+b\mat{0\\1\\-1}
		=t^3\mat{1\\-1\\0}+s\mat{1\\1\\-2}=\vec q(t,s).
	\]
	If we can find a formula for $a$ and $b$ in terms of $t$ and $s$, we can write down
	a formula for $\vec w$.

	From the vector equation above, we get the following system of equations.
	\begin{align*}
		a &= t^3+s\\
		-a+b&=-t^3+s\\
		-b&=-2s
	\end{align*}
	We conclude that $a=t^3+s$ and $b=2s$.  Therefore,
	\[
		\vec w(t,s) = (t^3+s, 2s).
	\]
\end{example}

\subsection{Isometric Parameterizations}

When parameterizing curves, arc-length parameterizations stands out as special.  If
you have an arc-length parameterization of a curve, line integrals become easy, speed is always one,
and you can imagine the curve as the real line picked up and bent, but \emph{not stretched}, and
placed into space.  The higher-dimensional analog of
an arc-length parameterization is an \emph{isometric parameterization}\index{isometric
parameterization}.

\begin{definition}[Isometric Parameterization]
	Let $\vec p:\R^n\to\mathcal S$ be a parameterization of $\mathcal S$.  The parameterization
	$\vec p$ is called an \emph{isometric parameterization} if the speed of $\vec p$ with
	respect to each parameter is one and $\vec p$ preserves area.
\end{definition}

Suppose $\mathcal S$ is a surface (that is, a two dimensional object) and that $\vec p:\R^2\to\mathcal S$
is an isometric parameterization.  We can then view $\mathcal S$ as the plane $\R^2$ picked up, bent, 
but \emph{not stretched}, and placed into space.  Any calculations you might want to do on 
the surface $\mathcal S$ can be done equally well by using $\vec p$ to convert them to calculations
on $\R^2$.

\begin{example}
	Show that $\vec p:\R^2\to\R^2$ given by $\vec p(t,s)=\tfrac{t}{\sqrt{2}}\mat{1\\1}+\tfrac{s}{\sqrt{2}}\mat{1\\-1}$
	is an isometric parameterization of $\R^2$.

	To show that $\vec p$ is an isometric parameterization of $\R^2$, we need to show that the speed with respect
	to each parameter is one and that it preserves area.  We'll verify the speed first.
	Let $t_0$ and $s_0$ be fixed and consider
	\begin{align*}
		\text{speed with respect to $t$} &= \lim_{h\to 0} 
		\frac{\norm*{\left(\tfrac{t_0+h}{\sqrt{2}} \mat{1\\1} + \tfrac{s_0}{\sqrt{2}}\mat{1\\-1}\right)
		-\left(\tfrac{t_0}{\sqrt{2}} \mat{1\\1} + \tfrac{s_0}{\sqrt{2}}\mat{1\\-1}\right)}
		}{\abs{h}}\\
		&= \frac{\abs*{\tfrac{h}{\sqrt{2}}}}{\abs{h}}\norm*{\mat{1\\1}} = 1.
	\end{align*}
	Computing the speed with respect to $s$ is similar and, indeed, the speed with
	respect to $s$ is $1$.

	Now, to verify that $\vec p$ preserves area, we only need to consider the image
	of rectangles under $\vec p$.\footnote{
		Using rectangles, we can decompose an area into the limit of the sum of a bunch of
		rectangles}
	Recall that $[t,t+\Delta t]\times[s,s+\Delta s]\subseteq\R^2$ is a rectangle with 
	lower left corner at $(t,s)$ and sides of length $\Delta t$ and $\Delta s$.
	Thus, we only need to verify
	that $\vec p([t,t+\Delta t]\times[s,s+\Delta s])$ has area $\Delta t\Delta s$.

	Graphing, we see that the corners of $\vec p([t,t+\Delta t]\times[s,s+\Delta s])$ are at

	XXX Finish
\end{example}

\begin{example}
	Find an isometric parameterization the cylinder with radius one and height one centered at
	the origin in $\R^3$ and oriented with height along the $z$-axis.

	XXX Finish
\end{example}

Isometric parameterizations are really neat, and as we'll see in the future, if we have an isometric
parameterization, many computations become easier.  Alas, though for most curves there exists
an arc-length parameterization, very few multi-dimensional objects have isometric parameterizations.
If an object has \emph{intrinsic} curvature, like a sphere, there could never be an isometric parameterization.
Asking that both speed and area be preserved is just too much.

However, there are several other special types of parameterizations for multiple dimensional objects.

\begin{definition}[Conformal Mapping]
	A parameterization $\vec p:\R^n\to\mathcal S$ is called a \emph{conformal map}\footnote{ \emph{Map}
	is just another word for function.}
	if it preserves angles.
\end{definition}

\begin{definition}[Volume Preserving Transformation]
	A parameterization $\vec p:\R^n\to\mathcal S$ is called a \emph{volume-preserving transformation}\footnote{ 
	\emph{Transformation} is just another word for function} if $\vec p$ preserves volume.
\end{definition}

Every isometric parameterization is both a conformal mapping and a volume-preserving transformation, but
there can be conformal mappings which aren't volume-preserving or isometric and volume-preserving transformations
which aren't conformal or isometric.  We won't study these types of parameterizations in detail, but
they show up in the domains of \emph{complex analysis} and the study of \emph{general relativity}.


\begin{exercises}
\end{exercises}

\section{Coordinate Systems}

Coordinate systems are an important special case of multi-dimensional parameterizations.
In fact, any parameterization $\vec p:\R^n\to\R^n$ gives rise to a coordinate system.  Rather than
starting off with a theoretical discussion of coordinate systems,
let's look at an
important coordinate system for $\R^2$.

\subsection{Polar Coordinates}
Polar coordinates\index{polar coordinates} describe points in $\R^2$ by a \emph{distance}, denoted $r$,
and an \emph{angle}, denoted $\theta$.  A point $\vec x\in\R^2$ specified by the polar-coordinates
$(r,\theta)$ is the point at distance $r$ from the origin at an angle of $\theta$
measured counter clockwise from the $x$-axis.

\begin{center}
	\newcommand{\tikzAngleOfLine}{\tikz@AngleOfLine}                               
	  \def\tikz@AngleOfLine(#1)(#2)#3{%                                            
	  \pgfmathanglebetweenpoints{%                                                 
	    \pgfpointanchor{#1}{center}}{%                                             
	    \pgfpointanchor{#2}{center}}                                               
	  \pgfmathsetmacro{#3}{\pgfmathresult}%                                        
	  }                                                                            
	\newcommand{\tikzMarkAngle}[3]{                                                
	\tikzAngleOfLine#1#2{\AngleStart}                                              
	\tikzAngleOfLine#1#3{\AngleEnd}                                                
	\draw #1+(\AngleStart:0.55cm) arc (\AngleStart:\AngleEnd:0.55cm);              
	} 

	
\begin{tikzpicture}
		\begin{axis}[
		    anchor=origin,
		    disabledatascaling,
		    xmin=-1,xmax=5,
		    ymin=-1,ymax=3,
		    x=1cm,y=1cm,
		    grid=both,
		    grid style={line width=.1pt, draw=gray!10},
		    %major grid style={line width=.2pt,draw=gray!50},
		    axis lines=middle,
		    minor tick num=0,
		    enlargelimits={abs=0.5},
		    axis line style={latex-latex},
		    ticklabel style={font=\tiny,fill=white},
		    xlabel style={at={(ticklabel* cs:1)},anchor=north west},
		    ylabel style={at={(ticklabel* cs:1)},anchor=south west},
xticklabels={,,},
yticklabels={,,}
		]

	\coordinate (A) at (2.5,2);
	\coordinate (X) at (-.07,.07);
		\draw[,myred!60!white,dashed] (A) -- (0,0) ;
			\draw [mypink,fill] (A) circle (1.5pt) node [ right] {$(r,\theta)$};
	
		\draw[decoration={brace, mirror}, decorate] ($(A)+(X)$) -- ($(0,0) +(X)$)
			node [midway,above left] {$r$};
\coordinate (O) at (0,0);
\coordinate (B) at (1,0);

		\node[] at (.75,.23) {$\theta$};
		\end{axis}
	\tikzMarkAngle{(O)}{(A)}{(B)};
	\end{tikzpicture}
\end{center}

Using trigonometry, we can deduce the rectangular coordinates of a point in polar coordinates.
In particular,
\begin{equation}
	\label{EQPOLAR}
	x=r\cos\theta\qquad\text{and}\qquad y=\sin\theta.
\end{equation}
We can also write the polar coordinates of a point $\vec x=(x,y)$ from its rectangular coordinates via
\[
	r=\sqrt{x^2+y^2}\qquad\text{and}\qquad \theta=\arctan\left( \tfrac{y}{x}\right).
\]

Typically when we use polar coordinates, we assume $r\in[0,\infty)$ and $\theta\in[0,2\pi)$, however
Equations \eqref{EQPOLAR} are valid for all $r,\theta\in\R$, and so we will let
$r$ and $\theta$ take non-traditional values when it suits us.

There are a couple of things to notice about polar coordinates.  First, if you list
the coordinates in the order $(r,\theta)$, then polar coordinates form a right-handed
coordinate system\footnote{ We haven't defined what it means for a coordinate system
whose ``axes'' are curved to be right-handed.  For polar coordinates it means that if
you take velocity vectors for the $r$ direction and the $\theta$ direction, the $\theta$
velocity vector is oriented counter clockwise compared to the $r$ velocity vector.  A more
appropriate name for this would be \emph{locally} right handed.}. 

\begin{center}
	\newcommand{\tikzAngleOfLine}{\tikz@AngleOfLine}                               
	  \def\tikz@AngleOfLine(#1)(#2)#3{%                                            
	  \pgfmathanglebetweenpoints{%                                                 
	    \pgfpointanchor{#1}{center}}{%                                             
	    \pgfpointanchor{#2}{center}}                                               
	  \pgfmathsetmacro{#3}{\pgfmathresult}%                                        
	  }                                                                            
	\newcommand{\tikzMarkAngle}[3]{                                                
	\tikzAngleOfLine#1#2{\AngleStart}                                              
	\tikzAngleOfLine#1#3{\AngleEnd}                                                
	\draw #1+(\AngleStart:0.55cm) arc (\AngleStart:\AngleEnd:0.55cm);              
	} 

	
\begin{tikzpicture}
		\begin{axis}[
		    anchor=origin,
		    disabledatascaling,
		    xmin=-1,xmax=5,
		    ymin=-1,ymax=3,
		    x=1cm,y=1cm,
		    grid=both,
		    grid style={line width=.1pt, draw=gray!10},
		    %major grid style={line width=.2pt,draw=gray!50},
		    axis lines=middle,
		    minor tick num=0,
		    enlargelimits={abs=0.5},
		    axis line style={latex-latex},
		    ticklabel style={font=\tiny,fill=white},
		    xlabel style={at={(ticklabel* cs:1)},anchor=north west},
		    ylabel style={at={(ticklabel* cs:1)},anchor=south west},
xticklabels={,,},
yticklabels={,,}
		]

	\coordinate (A) at (2.5,2);
\coordinate (MA) at (-2, 2.5);
	\coordinate (X) at (-.07,.07);
		\draw [mypink,fill] (A) circle (1.5pt) node [ right] {};
	
		\draw[,myred!60!white,dashed] (A) -- (0,0)node [midway,above left, black] {$r$} ;

	\draw[->,thick,mypink] (A) -- ($(A)+ 1/sqrt(2.5^2+2^2)*(A)$)node[midway, right, xshift=.2cm] {$r$ direction};
\draw[->,thick, myred!60!white] (A) -- ($(A)+ 1/sqrt(2.5^2+2^2)*(MA)$) node[midway, left,yshift=-.07cm] {$\theta$ direction};
\coordinate (O) at (0,0);
\coordinate (B) at (1,0);

		\node[] at (.75,.23) {$\theta$};
		\end{axis}
	\tikzMarkAngle{(O)}{(A)}{(B)};
	\end{tikzpicture}
\end{center}

Secondly, polar coordinates are not unique.  If we allow $r$ to take negative
values, the point described by $(r,\theta)=(1,\pi)$ is the same point described by $(r,\theta)=(-1,0)$.  
Most non-uniqueness issues are solved by insisting
$0\leq r <\infty$ and $0\leq\theta < 2\pi$.  However, even then 
$(r,\theta)=(0,\theta_0)$ describes the same point (the origin) no matter
the value of $\theta_0$.  For this reason,
we call the origin in polar coordinates a \emph{singularity}\footnote{ In mathematics, the word
\emph{singularity} is used to describe a point where ``bad things'' happen.  Ironically, the
singularity of polar coordinates is the place where there \emph{isn't} a singular description
in terms of polar coordinates.}.

\bigskip
Shapes with radial symmetry can often be described more easily in polar coordinates than
in rectangular.  For instance, let $\mathcal C$ be the circle of radius $7$ centered at the origin.
$\mathcal C$ can be described as
$\mathcal C=\Set{(r,\theta)\given r=7}$ in polar coordinates.  This doesn't have a dramatic
advantage over the representation $\mathcal C=\Set{(x,y)\given x^2+y^2=7^2}$ until you try to write
$\mathcal C$ as the graph of a function.  $\mathcal C$ cannot be described as the graph of a function
in rectangular coordinates, but it is the graph of the function $r=7$ in polar coordinates.

Circles and polar coordinates go together hand in hand.  Even circles centered away from the origin
have nice descriptions in polar coordinates.
\begin{example}
	Graph the function whose equation is given in polar coordinates by $r=\sin\theta$.

	XXX Finish
\end{example}

\subsection{Multiple Perspectives on Coordinate Systems}

We now have two coordinate systems for $\R^2$ under our belt: rectangular and polar.  
It's time to get philosophical so that we avoid getting confused later on.

Recall that we've been using $\R^2$ to represent two-dimensional Euclidean space---the plane.
We also use $\R^2$ to represent \emph{pairs of real numbers}.  If we were really pedantic,
we would use a different symbol for these two concepts because they really are different things.
However, once we draw $x$ and $y$ axes in the Euclidean plane, we see that pairs of real numbers
are \emph{equivalent} to points in the Euclidean plane.  That is, given any point in the Euclidean
plane, we can assign it exactly one pair of real numbers and any pair of real numbers corresponds
to exactly one point in the Euclidean plane.


This justified our use of $\R^2$ for both pairs of real numbers and the Euclidean plane.  However, our
situation now is more complicated.  Polar coordinates are \emph{also} pairs of real numbers
and polar coordinates \emph{also} describe points in the Euclidean plane.  Further, consider the function
$\vec p:\R^2\to\R^2$ defined by
\begin{equation}
	\label{EQPOLARPARAM}
	\vec p(r,\theta) = (r\cos\theta,r\sin\theta).
\end{equation}
On the one hand, $\vec p$ can be interpreted as a function that inputs polar coordinates and outputs
rectangular coordinates.  Under this interpretation, $\vec p$ \emph{doesn't do anything} to
the Euclidean plane.  On the other hand, we could interpret $\vec p$ as a function that takes in
rectangular coordinates and outputs rectangular coordinates.  Remember, $r$ and $\theta$ in Equation
\eqref{EQPOLARPARAM} are dummy variables.  They only have meaning insofar as they allow us to make
sense of the definition.  But we could have just as easily defined $\vec p$ by $\vec p(x,y) = (x\cos y,x\sin y)$.
It would have made no difference.  Under this interpretation $\vec p$ twists and warps the Euclidean
plane significantly.

When we start writing functions down in multiple coordinate systems, we need to be careful
that we know how to interpret these functions.  There are systems of notation what would allow
us to deal with multiple coordinate systems in an unambiguous manner, but we will take a more
pragmatic approach and have variable names carry extra meaning when talking about multiple coordinate systems.

For example, when we defined polar coordinates, we introduced the parameters of polar coordinates
as $r$ and $\theta$.  Therefore, if we write an equation using those variables,
for example $r=\sin2\theta$, that equation should be interpreted as a relationship in polar coordinates
and given a point $\vec x\in\R^2$, if we write $\vec x=(r,\theta)$ we interpret the description of $\vec x$
as given in polar coordinates.  Similarly, if we write $\vec x=(x,y)$ we interpret the description of $\vec x$
in rectangular coordinates.  This can lead to unfortunate-looking equations like $(r,\theta)=\vec x=(x,y)$.
This equation would be more clearly expressed as
\[
	(r,\theta)_{\text{polar}} = \vec x=(x,y)_{\text{rectangular}},
\]
but we will be sloppy if the context allows.  Other times we will just say in words, for example, ``the
point $(a,b)$ interpreted in polar coordinates,'' to make it clear what we mean.

There are also times when we want to interpret $(r,\theta)$ as a pair of real numbers instead
of as describing a point in Euclidean space.  In this situation, we may talk about the
point $(r,\theta)$ in the $r\theta$-plane.  This is code for, ``interpret $(r,\theta)$
as a pair of numbers that describe the rectangular coordinates of a point in the plane.''
However, by distinguishing the $r\theta$-plane from the usual $xy$-plane, we can draw helpful distinctions.

For example, recall $\vec p$ from earlier defined by $\vec p(r,\theta)=(r\cos\theta,r\sin\theta)$.  The function
$\vec p$ converts from polar coordinates to rectangular coordinates, but will also interpret it as
mapping the $r\theta$-plane to the $xy$-plane.

XXX Figure

While the graph of $r=2\theta$ is a line in the $r\theta$-plane, it is a spiral in the $xy$-plane.

XXX Figure

Despite this discussion, expect coordinate systems to occasionally be confusing.  Still,
their usefulness in applications outweighs the inconvenience of being confused.  We will examine
a couple more of the most popular non-rectangular coordinate systems.

\subsection{Cylindrical Coordinates}
Cylindrical coordinates\index{cylindrical coordinates} is a coordinate system for $\R^3$
that arises as a straightforward extension of polar coordinates.  Every point in $\R^3$ is
described by three numbers denoted by $r$, $\theta$, and $z$.

Recall that $\R^3$ can be described as $\R^2\times\R$.  That is, any point in $\R^3$ can be described
as a point in $\R^2$ coupled with a ``$z$-height.''  Cylindrical coordinates are obtained by
writing points in $\R^2$ in polar coordinates and then adding a $z$ component.

XXX Figure

It's related to rectangular coordinates by the formulas
\[
	x=r\cos\theta\qquad
	y=r\sin\theta\qquad
	z=z
\]
where $(r,\theta,z)$ represents a point in cylindrical coordinates and $(x,y,z)$ is the same point in rectangular
coordinates.



\begin{example}  $r = a$ describes an (infinite) cylinder
of radius $a$ centered on the $z$-axis.  If we let $a$ vary, we
obtain an infinite family of concentric cylinders.  We can treat
the case $a = 0$ (the $z$-axis) as a degenerate cylinder of
radius 0.

XXX Figure
\end{example}

\begin{example}$\theta = \alpha$ describes a \emph{half
plane} making angle $\alpha$ with the positive $xz$-plane.
In this half plane, $r$ can assume any non-negative
value and $z$ can assume any value.

XXX Figure
\end{example}
\begin{example}  $z = mr$  describes an (infinite)
  cone centered
on the $z$-axis with vertex at the origin.  For a fixed value
of $\theta$, we obtain a ray in this cone which starts at
the origin and extends to infinity.  This ray makes angle
$\arctan m$ with the $z$-axis, and if we let $\theta$
vary, the ray rotates around the $z$-axis generating the
cone.  Note also that if $m > 0$, the angle with the $z$-axis
is acute and the cone lies above the $x,y$-plane.  If
$m < 0$, the angle is obtuse, and the cone lies below the
$xy$-plane.  The case $m = 0$ yields the $x,y$-plane
($z = 0$) which may be considered a special `cone'.

Note that in rectangular coordinates,  $z = mr$ becomes
$z = m\sqrt{x^2 + y^2}$.

XXX Figure
\end{example}

\begin{example}  $r^2 + z^2 = a^2$ describes a sphere
of radius $a$ centered at the origin.  The easiest way to
see this is to put $r^2 = x^2 + y^2$ whence the equation
becomes $x^2 + y^2 + z^2 = a^2$.  The top hemisphere of the
sphere would be described by $z = \sqrt{a^2 - r^2}$
and the bottom hemisphere by
 $z = -\sqrt{a^2 - r^2}$.
\end{example}


\subsection{Spherical Coordinates}

Cylindrical coordinates are one way to generalize polar coordinates
to $\R^3$, but there is another way that is more useful for
problems with spherical symmetry. 
\emph{Spherical
coordinates}
\index{spherical coordinates}
\index{coordinates, spherical}
are denoted with the variables $\rho$, $\phi$, and $\theta$.

A point $P$ in space is defined by the coordinates $(\rho,\theta,\phi)$ as follows.   
The coordinate $\rho$ gives the distance
$\norm{\overrightarrow{OP}}$ of the point
to the origin.  It is always non-negative, and it should be
distinguished from the cylindrical coordinate $r$ which is the
distance from the $z$-axis.  The coordinate $\phi$ gives
the \emph{azimuthal angle}\index{azimuthal angle}, which
is the angle of declination between $\overrightarrow{OP}$ and the positive $z$-axis.
We will assume $\phi\in[0,\pi]$.  Finally, the coordinate $\theta$
denotes the \emph{longitudinal angle}\index{longitudinal angle},
and is the same as the $\theta$ from cylindrical coordinates.
Again, we assume $\theta\in[0,2\pi)$.

XXX Figure of spherical coords

Note the reason we restrict $\phi$ to the interval $[0,\pi]$ (rather than
$[0,2\pi)$. Fix $\rho$
and $\theta$.   If $\phi = 0$,
the point is on the positive $z$-axis, and, as $\phi$ increases,
the point swings down toward the negative $z$-axis. However, it stays
in the half plane determined by that value of $\theta$.  For
$\phi = \pi$, the point is on the negative $z$-axis, but if we
allow $\phi$ to increase further, the point swings into the
\emph{opposite} half plane with longitudinal angle $\theta + \pi$.
Such points can be obtained just as well by swinging down from
the positive $z$-axis in the opposite half plane determined by
$\theta + \pi$.

XXX Figure

The following relationships hold between spherical coordinates,
cylindrical coordinates, and rectangular coordinates. 
\begin{align*}
	r & = \rho \sin\phi \\
	z & = \rho \cos\phi \\
	\intertext{so}
	x &=  \rho \sin\phi \cos\theta \\
	y &=  \rho \sin\phi \sin\theta \\
	z & = \rho \cos\phi\\
	\intertext{and}
	\rho &= \sqrt{r^2 + z^2} = \sqrt{x^2 + y^2 + z^2} \\
	\tan\phi &= \frac rz\qquad\text{if } z \ne 0.
\end{align*}

It is important to note that unlike rectangular, polar, or cylindrical coordinates,
mathematicians and physicists use two different conventions for spherical coordinates.
We have introduced spherical coordinates in the order $(\rho,\theta,\phi)$.  This
paints spherical coordinates as an extension of polar coordinates in the $xy$-plane by
the addition of a third coordinate, $\phi$.  However, ordered this way, spherical
coordinates produce a \emph{left-handed} coordinate system.  Since physicists prefer
right-handed coordinate systems, they tend to use the order $(\rho,\phi,\theta)$ when
describing points in spherical coordinates\footnote{ Mathematics and physicists
differ in their conventions in several other regards.  Often times a force that is negative
to a physicist is positive to a mathematician and mathematicians tend to reverse the temperature
scale when doing thermodynamics.  All of the theorems come out the same, of course,
but it takes some interpretation to understand theorems from another field.}.  
We will primarily use the mathematical convention,
but when you read problems presented in spherical coordinates in other contexts, be
aware of what convention the author is using!


\begin{example}
$\rho = a$ describes a sphere of radius $a$ centered at the origin.
\end{example}

\begin{example}
$\phi = \alpha$ describes a cone making angle $\alpha$ with the
positive $z$-axis.  If $\phi<\pi/2$ the cone lies above the $xy$-plane;
	if $\phi>\pi/2$, the cone lies below the $xy$-plane; 
	and if $\phi=\pi/2$, the (degenerate) cone \emph{is} the $xy$-plane.
\end{example}

\begin{example}
$\theta = \beta$ describes a half plane starting from the $z$-axis
as before.
\end{example}

\begin{example}
$\rho = 2a\cos\phi$ describes a sphere of radius
$a$ centered at $(0,0,a)$.  You can see this by looking at
the half plane determined by fixing $\theta$.  In that half
plane, the locus is the \emph{semi-circle} with the given
radius and center.  If we then let $\theta$ vary, the effect is
to rotate the semi-circle about the $z$-axis and generate the
sphere

XXX Figure
\end{example}

If we fix $\rho = a$, we obtain a sphere of radius $a$.
Then $(\phi, \theta)$ specify the position of a point on
that sphere.  

For $\theta = $ constant, we obtain the semi-circle
which is the intersection of the half plane for that $\theta$
with the sphere.  That circle is called a {\it meridian of
longitude}.  This is exactly the concept of longitude used to
\index{meridian of longitude}
\index{longitude}
measure position on the surface of the Earth, except that we
use radians instead of degrees.  Earth's longitude is usually
measured in degrees east or west of the Greenwich Meridian.
That corresponds in our case to the positive and negative
directions from the 0-meridian.  

XXX Figure

For $\phi = $ constant, we obtain the circle which is the
intersection of the cone for that $\phi$ with the sphere.
Such circles are called {\it circles of latitude}.
\index{circle of latitude}
\index{latitude}
The coordinate $\phi$ is related to the notion of latitude on the surface
of the Earth, except that the latter is an angle
in degrees north or south of the \emph{equatorial plane}.  The
spherical coordinate $\phi$ is sometimes called \emph{co-latitude},
and we have $\phi = \pi/2 - \lambda$, where $\lambda$ is the latitude
measures from the equatorial plane (assuming both $\lambda$ and $\phi$ are
measured in radians).
The unique point with $\phi = 0$ is called the
\emph{north pole}, that with $\phi = \pi$ is called the
\emph{south pole}, and at the  poles $\theta$ is not well defined.

XXX Figure

\subsection{Coordinate Systems from Parameterizations}
We've looked at several important coordinate systems for $\R^2$ and $\R^3$,
but any parameterization of $\R^2$ or $\R^3$ gives rise to a coordinate system.

Consider $\vec p:\R^2\to\R^2$ given by $\vec p(a,b)=(a+b,a-b)$.  

\begin{exercise}
	Show that $\vec p:\R^2\to\R^2$ given by $\vec p(a,b)=(a+b,a-b)$ is
	a parameterization.  That is, that it is a continuous 
	one-to-one and onto function.
\end{exercise}

Geometrically, we can view $\vec p$ as taking the plane, rotating it by $45^\circ$,
and then
stretching it uniformly in all directions by a factor of $\sqrt{2}$.

XXX Figure

Alternatively, we can imagine that $\vec p$ describes how to take points described
in ``$ab$-coordinates'' and rewrite them in rectangular coordinates.  Call this new coordinate
system $\mathcal A$. We then get that
$\mathcal A$ coordinates relate to rectangular coordinates by
\[
	x=a+b\qquad\text{and}\qquad y=a-b.
\]
Drawing the lines $a=0$ and $b=0$ we see that $\mathcal A$ coordinates are very similar to
rectangular coordinates but with rotated and scaled axes.

XXX Figure

There might be good reason to use $\mathcal A$ coordinates, depending on your application.
For example, suppose you were working on a construction project and you are using rectangular
coordinates $(x,y)$ to represent the position north and east from the origin of your construction
lot.  If you're building a diagonal building, it might be nicer to describe locations in your building
using $\mathcal A$ coordinates.

\bigskip

Now, we might be tempted to propose that all coordinate systems come from parameterizations,
and this is \emph{almost} true.  However, coordinate systems like polar, cylindrical, and spherical
all have \emph{singularities}.  That is, there are points in space that are not uniquely
describable in those coordinate systems.  Thus, if we consider, for example, the function
$\vec p:\R^2\to\R^2$ which converts polar coordinates to rectangular coordinates, it isn't a true
parameterization because it isn't one-to-one at the origin.  We won't let this detail bother us---polar,
cylindrical, and spherical coordinates are still useful and they \emph{almost} come from parameterizations.
Occasionally we will even slip up and proclaim that $\R^2$ is parameterized by polar coordinates
or that $\R^3$ is parameterized by spherical coordinates\footnote{
	To a novice mathematician, it can be hard to discern the 
	patterns in what determines where a professional mathematician allows
	herself to be sloppy and where she maintains excruciating logical precision.
	Rest assured, after enough mistakes, one learns where one needs to tread carefully
	and where one can be more lax.
}.


\begin{exercises}
\end{exercises}
